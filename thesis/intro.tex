%%%%%%%%%%%%%%%%%%%%%%%%%%%%%%%%%%%%%%%%%%%%%%%%%%%%%%%%%%%%%%%%%%%%
% Einleitung
%%%%%%%%%%%%%%%%%%%%%%%%%%%%%%%%%%%%%%%%%%%%%%%%%%%%%%%%%%%%%%%%%%%%

\chapter{Introduction}\label{Introduction}
Climate change is one of the greatest challenges facing humanity today. Its potential impacts range from agricultural disruption that could lead to food insecurity, health impacts resulting from heatwaves and the spread of disease to socio-economic upheaval due to resource scarcity and population displacement \cite{parmesan2022climate}. The main driver of climate change is the emission of greenhouse gases, especially carbon dioxide (CO2) \cite{mikhaylov2020global}. The generation of electricity and heat was responsible for about 31~\% of all greenhouse gas emissions worldwide, and 43~\% of CO2 emissions in 2020 \cite{owidemissionsbysector}.

In addition, urbanization and transitions such as the electrification of transport cause increasing energy consumption \cite{vazquez2020citylearn, owidenergyproductionconsumption}, further emphasizing the need for low-carbon energy systems. The growing share of renewable energy sources leads to uncertainty in energy production due to their volatile nature. Conventional energy systems require consumption to be synchronized with generation. Oversizing the energy infrastructure would be conceivable to secure supply, but this would involve huge investments \cite{vazquez2020citylearn, suberu2014energy}.

Demand shifting is a more efficient approach to manage energy consumption. It involves reallocating the energy demand to times of high availability of renewable energy and using energy storage systems. The objectives of demand shifting include reducing peak energy demand, lowering electricity costs, improving grid reliability and promoting the integration of renewable energy. The literature considers exemplary energy systems to optimize, such as domestic hot water, electric vehicles or storage devices used in building energy systems \cite{vazquez2019reinforcement}. We focus on buildings for demand-side management since they account for 30~\% of the world's final energy consumption \cite{iea2023tracking}. However, because electricity has a higher intrinsic value to consumers than its price, price increases are insufficient for demand shifting when the availability of renewable energy is low. Instead, an automation of the process is necessary \cite{vazquez2019reinforcement}.

Artificial intelligence can automate energy systems to optimize fossil fuel savings while ensuring consumer satisfaction using reinforcement learning (RL). This agent-based method can be customized to specific objectives by adjusting a reward function. Agents can be trained individually for each building or collectively for multiple buildings and can manage both energy storage and direct demand shifting. 

Many studies using RL focus on reducing energy costs or consumption, often combined with measuring user satisfaction. Furthermore, many methods use algorithms for single agents and thus only deal with the demand reduction of individual buildings. However, optimizing the interaction between multiple buildings and their mutual influence is a more realistic and challenging task \cite{vazquez2019reinforcement}.

Our research investigates how agents can utilize social information to improve the performance of multi-agent RL. The existing literature includes cooperative multi-agent systems in autonomous microgrids \cite{kofinas2018fuzzy}, a focus on balancing supply and demand \cite{zhang2017deep}, and efforts to reduce peak demand through electric vehicles \cite{jiang2018multiple}. Reducing the peak demand is often the objective in multi-agent studies \cite{marinescu2015p, dusparic2015maximizing}, contrary to our goal of fossil savings. Algorithms such as Q-learning are widely used \cite{tan2018fast, jiang2011dynamic}. However, the traditional forms are unsuitable for continuous action spaces, leading us to consider actor-critic methods applied in contexts such as microgrids and production systems \cite{nakabi2021deep, lu2020multi}.

Psychological theories suggest that some human behaviors can be modeled using RL algorithms \cite{jara2019theory, lee2021joint}. More specifically, there are several theories, for example, decision biasing (DB) or value shaping, for modeling how study participants incorporate information into their decisions through demonstrators \cite{najar2020actions, selbing2014demonstrator, toyokawa2019social, witt2023social}. As far as we know, these theories only consider discrete action spaces and have yet to be applied in the context of demand management. Our work attempts to fill this gap by applying insights from psychology to the energy management domain and investigating the potential of social learning to improve the efficiency of RL algorithms in this field.

To summarize, our research focuses on reducing fossil fuel consumption in multiple buildings equipped with storage systems using multi-agent RL techniques. We aim to investigate how social information can assist in achieving this objective. Therefore, we do not aim to develop an algorithm that optimizes fossil fuel consumption.

Our work is structured as follows: 
Chapter 2 explains the necessary knowledge about RL and the Soft Actor-Critic (SAC) algorithm. We also discuss algorithm-independent methods for social learning and the Multi-Agent Reinforcement Learning with Iterative Sequential Action Selection (MARLISA) algorithm. 
Chapter 3 presents the CityLearn framework, which we use to model the environment and our adaptations. Moreover, we provide details on the data sets, key performance indicators (KPIs), and the training of the SAC baseline agents. We present their final performance and the demonstrators we use for the social methods. 
Chapter 4 presents these social methods in detail, describing our experiments, listing the results, and discussing them. 
In Chapter 5, we evaluate the method with the highest performance on a new dataset, discuss these results again, and present future research approaches.  
Finally, in Chapter 6, we summarize our approach and results.

