%%%%%%%%%%%%%%%%%%%%%%%%%%%%%%%%%%%%%%%%%%%%%%%%%%%%%%%%%%%%%%%%%%%%
% Einleitung
%%%%%%%%%%%%%%%%%%%%%%%%%%%%%%%%%%%%%%%%%%%%%%%%%%%%%%%%%%%%%%%%%%%%

\chapter{Introduction}\label{Introduction}

\todo[inline]{introduction, background (additional chapters), structure of thesis}

- Weltweiter Energieverbrauch steigt seit über 20 jahren \cite{energy2023statistical} zb wegen zunehmender urbanization und bspw electrikautos \cite{vazquez2020citylearn}
- Dies und der zunehmender anteil führt zu neuen herausforderung in der sicherung eines stabilen energienetzes \cite{energy2023statistical, vazquez2020citylearn} 
- 


- Rolling blackouts -> one solution oversize energy systems, but high investments neccessary [0]
- Increasing storage -> improve security of supply and reduce cost [0]
- RES -> introduce uncertainity in energy generation, shift demand to high availability [0], electrifiying only meaningful if energy generation is decarbonized, also mismatch between energy availability and demand [2]
- Often only single-agent considered [1]
- "When large amounts of energy storage are available, there is a risk of the energy consuming agents acting greedily by taking non-coordinated peak shifting actions that can lead to an overall peak shifting rather than peak reduction" [3]

Advantages RL
- Data-driven and model-free -> cost-effictive, plug-and-play [0]
Comparison to Rule-based Control (RBC) and model predictive control (MPC) - tutorial citylearn



Challenges in Energiesicherung:
- Transport electrification due to reducing environmental impact of transport \cite{zhang2020role}
- Global climate change -> low carbon energy systems, thus more renewable energy


Traditionell: Energie muss verbraucht werden wenn sie produziert wird --> gilt auch für erneuerbare 

"Incorporating Social Learning into Multi-Agent Reinforcement Learning to Lower Carbon Emissions in Energy Systems"


GENERELL: Lower Carbon Emissions in Energy Systems
- Climate change one of biggest human challenge (e.g. agricultural disruptions leading to food insecurity, health impacts from heatwaves and disease spread, socio-economic challenges due to resource scarcity and displacement of populations)
- Main driver of climate change: greenhouse gas emissions, particularly carbon dioxide (CO2)
- 2020 main source: electricity and heat (world-wide about 31 \% of all greenhouse, 43 \% of co2)
- Hence low-carbon energy systems important. at the same time increasing world wide energy consumption due to urbanization and e.g. transport electrificaton
- This in combination increases the share of renewable energy sources, which introduces more uncertainity in energy generation. traditionally, energy must be consumed when it is prodcued. 
- one solution would be to oversize energy systems thus always but for that high investmens neccessary
- more efficient: demand-shift and storage. hence, a shift of demand to time points where high availablity of renewable energy is given as well as introducing energy storage is of interest- the storage improves the security of energy systems and reduces costs.
- Goals of demand-shift: reduce peak energy demand, lower electricity costs, enhance grid reliability, and increase the integration and utilization of renewable energy sources

- often used for demand-side management: buildings are used because operations of buildings account for 30 \% of global final energy consumption 
- demand-side management must be automated because electricity is more valueable for consumer than the price --> just increasing the price when no renewable energy available is not sufficient
- usage of articial intelligence allows to automate energy systems with regard to fossil energy savings while still satisfy the consumer
- reinforcement learning is agent-based AI method that can be adapted to specific goals using a reward that the agents aims to maximize over time. either one agent per building is trained or one agent for multiple buildings e.g. in an small autonomous grid that manages all buildings. the agents determine either how the energy storage is used or directly shift the demand. 
- Relevant energy systems where you can improve are for example domestic hot water, electric vehicle or storage devices used in building energy systems
- Many publications focus reducing the energy cost or energy consumption, both also combined with the satisfaction. 
- many publications only use single-agent algorithms, so only focus on the demand-response of one single building. however, it is more realistic and importand to optimize the interaction of multiple buildings and how they may influence each other
- our intereset: how can the agents use social information to improve their performance
- \cite{kofinas2018fuzzy} cooperative multi-agent but only in a stand-alone micro-grid, \cite{zhang2017deep} focus on power balance between the supply and demand side, \cite{jiang2018multiple} focus on electrical vehicles with goal to lower peak demand, peak demand also \cite{marinescu2015p, dusparic2015maximizing}
- RL Algorithms often used are Q-Learning, e.g. \cite{tan2018fast, jiang2011dynamic} but off-the-shelve does not support continuous action spaces, so we focus on actor-critic as in \cite{nakabi2021deep} (only microgrid) \cite{lu2020multi} (only manufacturing systems)
- in psychology, there exist a theory that some human behaviours can be modeled as reinforcement algorithm \cite{jara2019theory, lee2021joint}. also behaviour of study participants when demonstrations are given is often modeled using reinforcement learning algorithms \cite{}

We focus on electricity and our goal is to lower carbon emissions storage devices used in building energy systems


Demand-Side Managem


METHODE: Reinforcement learning weil cool

FORSCHUNGSFRAGE: wie kann man da soziale information reinbringen und weniger fossile verbrauchen
