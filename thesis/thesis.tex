%%%%%%%%%%%%%%%%%%%%%%%%%%%%%%%%%%%%%%%%%%%%%%%%%%%%%%%%%%%%%%%%%%%%%%%%%%%%%
%%% LaTeX-Rahmen fuer das Erstellen von Masterarbeiten
%%%%%%%%%%%%%%%%%%%%%%%%%%%%%%%%%%%%%%%%%%%%%%%%%%%%%%%%%%%%%%%%%%%%%%%%%%%%%

%%%%%%%%%%%%%%%%%%%%%%%%%%%%%%%%%%%%%%%%%%%%%%%%%%%%%%%%%%%%%%%%%%%%%%%%%%%%%
%%% allgemeine Einstellungen
%%%%%%%%%%%%%%%%%%%%%%%%%%%%%%%%%%%%%%%%%%%%%%%%%%%%%%%%%%%%%%%%%%%%%%%%%%%%%

\documentclass[twoside,12pt,a4paper]{report}
%\usepackage{reportpage}
\usepackage{epsf}
\usepackage{graphics, graphicx}
\usepackage{latexsym}
\usepackage[margin=10pt,font=small,labelfont=bf]{caption}
\usepackage[utf8]{inputenc}
\usepackage[toc,page]{appendix}


% Own packages
\usepackage[colorinlistoftodos]{todonotes}
\usepackage{algorithm}
\usepackage{algpseudocode}
\usepackage{amssymb}
\usepackage{tabularx}
\usepackage{url}
\usepackage{amsmath}
\usepackage[permil]{overpic}
\usepackage{multirow}
\usepackage{enumitem}
\usepackage{fancyhdr}
\usepackage{booktabs} 

\newcolumntype{Y}{>{\centering\arraybackslash}X}
\newcommand\numberthis{\addtocounter{equation}{1}\tag{\theequation}}
\DeclareMathOperator*{\argmax}{arg\,max}
\DeclareMathOperator*{\argmin}{arg\,min}

\hyphenation{thatshouldnot}


\textwidth 14cm
\textheight 22cm
\topmargin 0.0cm
\evensidemargin 1cm
\oddsidemargin 1cm
%\footskip 2cm
\parskip0.5explus0.1exminus0.1ex

% Kann von Student auch nach pers\"onlichem Geschmack ver\"andert werden.
\pagestyle{headings}
\fancypagestyle{plain}{%
\fancyhf{} % clear all header and footer fields
\fancyhead{} % clear all header fields
\fancyfoot{} % clear all footer fields
\fancyhead[RO]{\thepage}
\fancyhead[LE]{\thepage}
\renewcommand{\headrulewidth}{0pt}
\renewcommand{\footrulewidth}{0pt}
}

\sloppy

\begin{document}

%%%%%%%%%%%%%%%%%%%%%%%%%%%%%%%%%%%%%%%%%%%%%%%%%%%%%%%%%%%%%%%%%%%%%%%%%%%%
%%% hier steht die neue Titelseite 
%%%%%%%%%%%%%%%%%%%%%%%%%%%%%%%%%%%%%%%%%%%%%%%%%%%%%%%%%%%%%%%%%%%%%%%%%%%%
 
\begin{titlepage}
 \begin{center}
  {\LARGE Eberhard Karls Universit\"at T\"ubingen}\\
  {\large Mathematisch-Naturwissenschaftliche Fakult\"at \\
Wilhelm-Schickard-Institut f\"ur Informatik\\[4cm]}
  {\huge Master Thesis Computer Science\\[2cm]}
  {\Large\bf  Incorporating Social Learning into Multi-Agent Reinforcement Learning
to \\Lower Carbon Emissions in Energy Systems\\[1.5cm]}
 {\large Kristina Lietz}\\[0.5cm]
01.12.2023\\[3.5cm]
{\small\bf Reviewers}\\[0.5cm]
  \parbox{7cm}{\begin{center}{\large Dr. Nicole Ludwig}\\
   (Machine Learning in Sustainable Energy Systems)\\
  {\footnotesize Exzellenzcluster "Maschinelles Lernen"\\
	Universit\"at T\"ubingen}\end{center}}\hfill\parbox{7cm}{\begin{center}
  {\large Prof. Setareh Maghsudi}\\
  (Decision Making)\\
  {\footnotesize Wilhelm-Schickard-Institut f\"ur Informatik\\
	Universit\"at T\"ubingen}\end{center}
 }
  \end{center}
\end{titlepage}

%%%%%%%%%%%%%%%%%%%%%%%%%%%%%%%%%%%%%%%%%%%%%%%%%%%%%%%%%%%%%%%%%%%%%%%%%%%%
%%% Titelr"uckseite: Bibliographische Angaben
%%%%%%%%%%%%%%%%%%%%%%%%%%%%%%%%%%%%%%%%%%%%%%%%%%%%%%%%%%%%%%%%%%%%%%%%%%%%

\thispagestyle{empty}
\vspace*{\fill}
\begin{minipage}{11.2cm}
\textbf{Lietz, Kristina:}\\
\emph{Incorporating Social Learning into Multi-Agent Reinforcement Learning
to Lower Carbon Emissions in Energy Systems}\\ Master Thesis Computer Science\\
Eberhard Karls Universit\"at T\"ubingen\\
Thesis period: 01.06.2023~--~01.12.2023
\end{minipage}
\newpage

%%%%%%%%%%%%%%%%%%%%%%%%%%%%%%%%%%%%%%%%%%%%%%%%%%%%%%%%%%%%%%%%%%%%%%%%%%%%

\pagenumbering{roman}
\setcounter{page}{1}

%%%%%%%%%%%%%%%%%%%%%%%%%%%%%%%%%%%%%%%%%%%%%%%%%%%%%%%%%%%%%%%%%%%%%%%%%%%%
%%% Seite I: Zusammenfassug, Danksagung
%%%%%%%%%%%%%%%%%%%%%%%%%%%%%%%%%%%%%%%%%%%%%%%%%%%%%%%%%%%%%%%%%%%%%%%%%%%%


\section*{Abstract}
Reducing the consumption of fossil fuels is crucial for mitigating climate change, as it reduces greenhouse gas emissions. At the same time, renewable energy creates more uncertainty in the energy supply, challenging the traditional approach of consuming available energy immediately.
This thesis explores integrating social learning into multi-agent reinforcement learning to minimize fossil fuel consumption in building energy management. By equipping each building with a battery, the study employs agents to learn optimal charging and discharging strategies. 
We use the CityLearn framework, modified to suit our requirements and define four KPIs for evaluation. As a baseline, we train a non-social Soft Actor-Critic (SAC) agent, which we compare to four social learning methods. These methods include imitation learning for policy mimicry, SAC-DemoPol, which biases the policy toward demonstrator actions, SAC-DemoQ, which increases the Q-values of demonstrator actions, and the MARLISA algorithm for collaborative fossil energy saving. Results indicate that the SAC-DemoQ agents, trained with a pre-trained demonstrator, outperform the SAC baseline by approximately 1.5 \% in fossil energy savings. Using novel building data, the social agents save about 1 \% more fossil energy. The other social methods achieved minimal or no improvement. We suggest future evaluation using more diverse datasets and employing methods such as value shaping, clustering for optimal demonstrator selection, and fine-tuning the imitation learning rate to enhance the efficacy of the proposed strategy in reducing fossil fuel consumption in buildings.

\newpage
\section*{Zusammenfassung}
Die Verringerung des Verbrauchs fossiler Brennstoffe und die damit verbundene Reduzierung der Treibhausgasemissionen sind für die Eindämmung des Klimawandels von entscheidender Bedeutung. Gleichzeitig schaffen erneuerbare Energien mehr Unsicherheit in der Energieversorgung, was den traditionellen Ansatz des sofortigen Verbrauchs der verfügbaren Energie infrage stellt.
In dieser Arbeit wird die Integration von sozialem Lernen in Multi-Agenten Reinforcement Learning untersucht, um den Verbrauch fossiler Brennstoffe im Energiemanagement von Gebäuden zu minimieren. Es werden Agenten entwickelt, welche optimale Lade- und Entladestrategien der Batterien von Gebäuden erlernen. 
Wir verwenden das CityLearn-Framework, das wir an unsere Anforderungen anpassen, und definieren vier KPIs für die Performanzbewertung. Als Ausgangsbasis trainieren wir einen nicht-sozialen Soft Actor-Critic (SAC)-Agenten, den wir mit vier sozialen Lernmethoden vergleichen. Zu diesen Methoden gehören das Nachahmungslernen für die Nachahmung der Policy eines Demonstrators, SAC-DemoPol, wobei die Policy in Richtung der Demonstrantenaktionen gelenkt wird, SAC-DemoQ, wobei die Q-Werte der Demonstrantenaktionen erhöht werden, und der MARLISA-Algorithmus für kollaboratives fossiles Energiesparen. Die Ergebnisse zeigen, dass die SAC-DemoQ-Agenten, die mit einem vortrainierten Demonstrator trainiert wurden, die unsozialen SAC-Agenten bei der Einsparung fossiler Energie um etwa 1,5 \% übertreffen. Bei Verwendung neuer Gebäudedaten sparen die sozialen Agenten etwa 1 \% mehr fossile Energie. Die anderen sozialen Ansätze zeigen keine oder nur minimale Verbesserung. Wir schlagen vor, zukünftige Evaluierungen mit vielfältigeren Datensätzen und unter Verwendung von Methoden wie Value Shaping, Clustering für die optimale Auswahl von Demonstratoren und Feinabstimmung der Imitationslernrate durchzuführen, um die Wirksamkeit der entwickelten Strategie bei der Reduzierung des Verbrauchs fossiler Brennstoffe in Gebäuden zu verbessern.



\newpage
\section*{Acknowledgements}
I sincerely thank all those who have supported me in my studies and during this work.

First, I would like to express my deepest gratitude to my closest family: My sister Annika, my parents Heike, Rüdiger, Andrea, and my grandma Ellen. Your emotional support and encouragement help guide me through all the difficult times and inspire me to persevere. Your belief in me and my abilities was a source of motivation and perseverance.

I am also immensely grateful to my friends who have supported me, especially in the last few months: Jule, Elli, Basti, Steffen, Fabio and Flop. Your support, camaraderie and understanding have been crucial in this work. The proofreading from some of you also helped me a lot. The moments we have shared and the moral support you have given me have been invaluable and have given me balance and a sense of belonging.

Special thanks to my supervisor, Dr. Nicole Ludwig, for her guidance, patience, and expertise. Her insightful feedback and constructive criticism have been instrumental in shaping my research and writing. 

I am deeply grateful to all of you. Your collective support has made this work possible. Thank you.

\cleardoublepage

%%%%%%%%%%%%%%%%%%%%%%%%%%%%%%%%%%%%%%%%%%%%%%%%%%%%%%%%%%%%%%%%%%%%%%%%%%%%%
%%% Table of Contents
%%%%%%%%%%%%%%%%%%%%%%%%%%%%%%%%%%%%%%%%%%%%%%%%%%%%%%%%%%%%%%%%%%%%%%%%%%%%%

\renewcommand{\baselinestretch}{1.3}
\small\normalsize

\setcounter{tocdepth}{1}
\tableofcontents

\renewcommand{\baselinestretch}{1}
\small\normalsize

\cleardoublepage

%%%%%%%%%%%%%%%%%%%%%%%%%%%%%%%%%%%%%%%%%%%%%%%%%%%%%%%%%%%%%%%%%%%%%%%%%%%%%
%%% List of Figures
%%%%%%%%%%%%%%%%%%%%%%%%%%%%%%%%%%%%%%%%%%%%%%%%%%%%%%%%%%%%%%%%%%%%%%%%%%%%%

\renewcommand{\baselinestretch}{1.3}
\small\normalsize

\addcontentsline{toc}{chapter}{List of Figures}
\listoffigures

\renewcommand{\baselinestretch}{1}
\small\normalsize

\cleardoublepage

%%%%%%%%%%%%%%%%%%%%%%%%%%%%%%%%%%%%%%%%%%%%%%%%%%%%%%%%%%%%%%%%%%%%%%%%%%%%%
%%% List of tables
%%%%%%%%%%%%%%%%%%%%%%%%%%%%%%%%%%%%%%%%%%%%%%%%%%%%%%%%%%%%%%%%%%%%%%%%%%%%%

\renewcommand{\baselinestretch}{1.3}
\small\normalsize

\addcontentsline{toc}{chapter}{List of Tables}
\listoftables

\renewcommand{\baselinestretch}{1}
\small\normalsize

\cleardoublepage

%%%%%%%%%%%%%%%%%%%%%%%%%%%%%%%%%%%%%%%%%%%%%%%%%%%%%%%%%%%%%%%%%%%%%%%%%%%%%
%%% List of abbreviations
%%%%%%%%%%%%%%%%%%%%%%%%%%%%%%%%%%%%%%%%%%%%%%%%%%%%%%%%%%%%%%%%%%%%%%%%%%%%%

% can be removed
\addcontentsline{toc}{chapter}{List of Abbreviations}
\chapter*{List of Abbreviations\markboth{LIST OF ABBREVIATIONS}{LIST OF ABBREVIATIONS}}
\begin{tabbing}
\textbf{FACTOTUM}\hspace{1cm}\=Schrott\kill
\textbf{CO2}\>Carbon Dioxide\\
\textbf{DB}\>Decision Biasing\\
\textbf{DHI}\>Diffuse Horizontal Irradiance \\
\textbf{DNI}\>Direct Normal Irradiation \\
\textbf{DDPG}\>Deep Deterministic Policy Gradient \\
\textbf{GHI}\>Global Horizontal Irradiance \\
\textbf{i.i.d.}\>independent and identically distributed \\
\textbf{ILR}\>Imitation Learning Rate \\
\textbf{KPI}\>Key performance indicator \\
\textbf{kWh}\>Kilowatt-hour \\
\textbf{L2 loss}\>Mean Squared Error Loss \\
\textbf{MARLISA}\> Multi-Agent Reinforcement Learning with Iterative \\ 
\>Sequential Action Selection \\
\textbf{MDP}\>Markov decision process \\
\textbf{NSRDB}\>National Solar Radiation Database \\
\textbf{NY}\> New York \\
\textbf{NYISO}\>New York Independent System Operator \\
\textbf{PCC}\>Pearson Correlation Coefficient \\
\textbf{PRB}\>Prioritized Replay Buffer \\
\textbf{PV}\> Photovoltaic \\
\textbf{RBC}\>Rule-Based Controller \\
\textbf{RL}\>Reinforcement learning \\
\textbf{SAC}\>Soft actor-critic \\
\textbf{TD error}\> Temporal Difference error \\
\end{tabbing}

\cleardoublepage


%%%%%%%%%%%%%%%%%%%%%%%%%%%%%%%%%%%%%%%%%%%%%%%%%%%%%%%%%%%%%%%%%%%%%%%%%%%%%
%%% Der Haupttext, ab hier mit arabischer Numerierung
%%% Mit \input{dateiname} werden die Datei `dateiname' eingebundent
%%%%%%%%%%%%%%%%%%%%%%%%%%%%%%%%%%%%%%%%%%%%%%%%%%%%%%%%%%%%%%%%%%%%%%%%%%%%%

\pagenumbering{arabic}
\setcounter{page}{1}

%% Introduction
%%%%%%%%%%%%%%%%%%%%%%%%%%%%%%%%%%%%%%%%%%%%%%%%%%%%%%%%%%%%%%%%%%%%
% Einleitung
%%%%%%%%%%%%%%%%%%%%%%%%%%%%%%%%%%%%%%%%%%%%%%%%%%%%%%%%%%%%%%%%%%%%

\chapter{Introduction}\label{Introduction}

\todo[inline]{introduction, background (additional chapters), structure of thesis}


\cleardoublepage

%% 
\input{Background}
\cleardoublepage

%% 
%%%%%%%%%%%%%%%%%%%%%%%%%%%%%%%%%%%%%%%%%%%%%%%%%%%%%%%%%%%%%%%%%%%%
% Grundlagen
%%%%%%%%%%%%%%%%%%%%%%%%%%%%%%%%%%%%%%%%%%%%%%%%%%%%%%%%%%%%%%%%%%%%

\subsection{SAC using demonstrator transitions}
\todo[inline]{Font in tabellen beschreibungen verkleinern}
As mentioned in Section \ref{sec:background-social-learning}, a widely used method to improve learning is sampling demonstrator transitions and storing them in a Prioritized Replay Buffer. These kind of buffer add priorities based on the Temporal difference (TD) error to the transitions: Transistion with a higher TD error have an higher priority, assuming that these are more difficult to learn and hence should be seen more often during training. Thus, transitions with a higher priority are sampled with a higher probability. The priorities are updated during the training process \cite{schaul2015prioritized}.

\todo[inline]{Present the buildings used in training and the demonstrator building (and why this)}
For implementing the SAC agent using the demonstrator transitions, we first trained one building with 

\subsection{\textcolor{red}{Social Agent I}}
\textcolor{red}{Policy update}
First: normal policy update

Then: policy update using demonstrator actions

Classical loss $l$ to minimize is is 
\begin{align*}
	l &= \mathbb{E}_{s_t \sim \mathcal{D}, \epsilon_t \sim \mathcal{N}}[\alpha \log\pi_{\phi}(f_{\phi}(\epsilon_t;s_t)|s_t)-Q_{\theta}(s_t, f_{\phi}(\epsilon_t;s_t))] \\ 
	&=  \mathbb{E}_{s_t \sim \mathcal{D}, \epsilon_t \sim \mathcal{N}}[\alpha \mathcal{H}_{SAC}-\mathcal{V}_{SAC}] \numberthis \label{eqn:policy-loss}
\end{align*}
with the entropy term that aims to maximize randomness $\mathcal{H}_{SAC}$ and the value term that aims to maximize the estimated Q-Value of the action $\mathcal{V}_{SAC}$. For incorporating demonstrator actions similar to decision biasing described in chapter \ref{sec:decision-biasing}, so increase the probability of observed demonstrator actions or increase the estimated Q-Value of these, we modify the terms as follows: In mode 1, we use the actions sampled from the demonstrator $f^d_{\Phi}(\epsilon_t;s_t)$ in the value term and increase the estimated Q-Value by adding a fraction of the absolute value of it using the imitation learning rate $\alpha_i$:
\begin{equation}
	\mathcal{V}_{M1} =  Q_{\theta}(s_t, f_{\phi}^d(\epsilon_t;s_t)) + \alpha_i|Q_{\theta}(s_t, f_{\phi}^d(\epsilon_t;s_t))|
\end{equation}
Also, the probability of taking the demonstrator action for the demonstrator is used in the entropy term:
\begin{equation}
	\mathcal{H}_{M1} =  \log\pi_{\phi}^d(f_{\phi}^d(\epsilon_t;s_t)|s_t)
\end{equation}
In mode 2, the probability of taking the demonstrator action is increased by the learning rate:
\begin{equation}
	\mathcal{H}_{M2} = \log\pi_{\phi}^d(f_{\phi}^d(\epsilon_t;s_t)|s_t) + \alpha_i|\log\pi_{\phi}^d(f_{\phi}^d(\epsilon_t;s_t)|s_t)|
\end{equation}
Mode 3 combines mode 1 and mode 2. Mode 4-6 are similar to the modes 1-3, but the probability of taking the demonstrator action in the current trained policy is used:
\begin{equation}
	\mathcal{H}_{M4} =  \log\pi_{\phi}(f_{\phi}^d(\epsilon_t;s_t)|s_t)
\end{equation}
\begin{equation}
	\mathcal{H}_{M5} = \log\pi_{\phi}(f_{\phi}^d(\epsilon_t;s_t)|s_t) + \alpha_i|\log\pi_{\phi}(f_{\phi}^d(\epsilon_t;s_t)|s_t)|
\end{equation}
All modes calculate the loss as defined in Equation \ref{eqn:policy-loss} using the defined entropy and value term. Table \ref{tab:social-policy-update-modes} summarizes the different modes and their entropy and value terms.

{
\renewcommand{\baselinestretch}{0.9} 
\normalsize
\begin{table}[htb]
\center
\begin{tabular}{|l|l|l|}
\hline
    \textbf{Mode} & \textbf{Entropy term} & \textbf{Value term} \\
  \hline\hline
  1 & $\mathcal{H}_{M1}$ &$\mathcal{V}_{M1}$
  \\ \hline
  2 & $\mathcal{H}_{M2}$& $\mathcal{V}_{SAC}$ \\ \hline
  3 & $\mathcal{H}_{M2}$& $\mathcal{V}_{M1}$ \\ \hline 
  4 & $\mathcal{H}_{M4}$ & $\mathcal{V}_{M1}$ \\ \hline
  5 & $\mathcal{H}_{M5}$ & $\mathcal{V}_{SAC}$ \\ \hline
  6 & $\mathcal{H}_{M5}$ &  $\mathcal{V}_{M1}$ \\ \hline 
\end{tabular}
  \caption[Modes for Social Agent focusing on the policy update]{Modes for Social Agent focusing on the policy update. Mode 1 increases the estimated Q-Value of the demonstrator action, mode 2 increases the probability of the demonstrator action and mode 3 combines both. Mode 4-6 are similar, but the probability of the demonstrator actions in the current trained policy is used instead of their probabilities in the demonstrator policy.}
  \label{tab:social-policy-update-modes}
\end{table}
}


\subsection{\textcolor{red}{Social Agent II}}
Since the social policy update did not improve the performance of the models, we decided to focus on the Q-Value update in the next step. For this, we also first conducted classical Q-Value and policy updates as defined in Algorithm \ref{alg:sac-update}. Next, we performed a social Q-Value update on both Q-Value networks where we defined the target values as follows:
\begin{equation}
	Q_{\bar{\theta}}(a_t^d,s_t) + \alpha_i|Q_{\bar{\theta}}(a_t^d,s_t)| 
\end{equation}
This follows the assumption of decision biasing, that actions taken by the demonstrator $a_t^d$ are good and should thus have a higher Q-Value as proposed by \cite{najar2020actions}. Afterwards, a soft update of the target networks parameters is performed, and, optionally, a standard policy update.





\cleardoublepage

%% 
%%%%%%%%%%%%%%%%%%%%%%%%%%%%%%%%%%%%%%%%%%%%%%%%%%%%%%%%%%%%%%%%%%%%
% Soziale Agenten
%%%%%%%%%%%%%%%%%%%%%%%%%%%%%%%%%%%%%%%%%%%%%%%%%%%%%%%%%%%%%%%%%%%%
\chapter{Social Agents}
   \label{sec:social-agents}
\noindent
\todo[inline]{Ziel dieses Kapitels ist eine Einf\"uhrung in die Thematik BlaBlaBla ...}

\section{Pretrained Demonstrator}
\label{sec:pretrained-demos}
For some of our methods, we used pre-trained demonstrators. We trained these demonstrators using the final hyperparameter values, reward function and state space as described in Section~\ref{sec:sac-baseline}. Also, we scaled the available renewable energy in the grid by the factor $k = 0.5$. We used two different buildings as pre-trained demonstrators, one among the training buildings and one not a training building.

\begin{figure}[htb]
\center
     \includegraphics[width=\textwidth]{figures/b5_b6_kpis.pdf}
  \caption{}
  \label{fig:b5-b6-kpis}
\end{figure}
We chose the buildings with the highest median PCC to the training buildings from the set of training buildings and the remaining ones: Building Five (training building) and Building Six. The resulting KPIs of both buildings are visualized in Figure~\ref{fig:b5-b6-kpis}. The values of the share of used PV and the fossil energy consumption are somewhat comparable to the (mean) performance of the baseline SAC agents when using all training buildings. However, the average daily renewable share, both from the grid and in total, is worse when using the battery. This effect could be because the energy demand from the grid of both buildings is lower when the battery is in use since more solar energy is utilized. As a result, the proportion of renewable energy used from the grid decreases because the solar generation of the buildings is correlated to the renewable production in the grid. This, in turn, has the same effect on the overall share of renewable energy used.


\section{SAC using Demonstrator Transitions}
\todo[inline]{Font in tabellen beschreibungen verkleinern}
As mentioned in Section~\ref{sec:background-social-learning}, a popular technique to improve learning is by sampling demonstration transitions and storing them in a prioritized replay buffer (PRB). This buffer assigns priorities to transitions based on their temporal difference (TD) error. Transitions with higher TD error are given higher priority because they are expected to be more challenging to learn and should be observed more frequently during training. The transitions with higher priority are more likely to be sampled. Priorities are updated during the training process \cite{schaul2015prioritized}.

To implement the SAC agent using the demonstrator transitions, we used the trained agents previously described in Section~\ref{sec:pretrained-demos}. These agents acted on the complete dataset hourly for an entire year. We saved these transitions and then trained the agents normally, as outlined in Section~\ref{sec:sac-baseline}. The only difference was that we filled the PRB with the demonstrator transitions at the beginning of the training for each agent. We used either the transitions from Building 5 or Building 6, but not both simultaneously. The results in comparison to the baseline SAC agents are visualized in Figure~\ref{fig:prb-kpis}.

\begin{figure}[htb]
\center
     \includegraphics[width=\textwidth]{figures/prb_kpis.pdf}
  \caption{}
  \label{fig:prb-kpis}
\end{figure}

The SAC agents using the demonstrator transitions perform worse than the baseline agents on all KPIs. However, using transitions from Building 5 resulted in better scores compared to the use of transitions from Building 6. This is consistent with the performance of both demonstrator buildings, as described in the previous chapter. It is possible that the better-performing Building 5 provides better guidance on how agents should behave.

The policy loss values observed during training increase for all agents instead of decreasing as expected. This is illustrated in Figure~\ref{fig:prb-losses} for Building 3 and the demonstrator transitions of Building 5, but it also applies to the other training buildings and demonstrator Building 6. This development suggests that with higher probability, actions with a low, possibly even negative, estimated Q-Value are taken. As a result, the agents are more likely to take 'bad' actions with greater certainty, ultimately leading to poorer performance.

\begin{figure}[htb]
\center
     \includegraphics[width=0.8\textwidth]{figures/prb_losses.pdf}
  \caption[Policy losses during training of Building 3]{Policy losses during training of Building 3 for the baseline SAC agent and the SAC agent when filling the PRB with the transitions of demonstrator Building 5 before training.}
  \label{fig:prb-losses}
\end{figure}


\section{Social Agent I}
In our first approach, based on decision biasing, we adjusted the loss function of the policy update. As described in Section~\ref{sec:decision-biasing}, decision biasing assumes that demonstrator actions are good and, therefore, the probability of choosing those actions or the Q-value of them should be increased. To accomplish this, we tested six different modes, all of which used an imitation learning rate that we also optimized. Also, we investigated whether it made a difference whether the demonstrated action was sampled from the demonstrator policy or the deterministic action (i.e., the learned mean) was used in one of the modes.

As a reminder, the classic policy objective function is defined as 
\begin{align*}
	J_{\pi}(\phi) &= \mathbb{E}_{s_t \sim \mathcal{D}, \epsilon_t \sim \mathcal{N}} \left [\alpha \log\pi_{\phi}(f_{\phi}(\epsilon_t;s_t)|s_t)-Q_{\theta}(s_t, f_{\phi}(\epsilon_t;s_t)) \right ] \\ 
	&=  \mathbb{E}_{s_t \sim \mathcal{D}, \epsilon_t \sim \mathcal{N}}[\alpha \mathcal{H}_{SAC}-\mathcal{V}_{SAC}] \numberthis \label{eqn:policy-loss}
\end{align*}
with the entropy term that aims to maximize randomness $\mathcal{H}_{SAC}$ and the value term that aims to maximize the estimated Q-Value of the action $\mathcal{V}_{SAC}$. 

We first trained the agents using the classic SAC algorithm in this social agent. For each gradient step, we incorporated an extra policy update step for each demonstrator $d \in D$, using the social objective function $J_{\pi}^{social}(\phi)$. This social objective comprises different entropy and value terms based on the mode but is fundamentally similar to the conventional objective. The training algorithm of \textcolor{red}{Social Agent I} is given in Appendix \ref{sec:app-algos}, Algorithm \ref{app:sac-social-1}. The only difference to the classic SAC algorithm are lines 14-16.

When operating in mode 1, we use the actions sampled from the demonstrator policy $a_t^d=f^d_{\phi}(\epsilon_t;s_t)$. The estimated Q-value in the value term is then increased by adding a fraction of the absolute value using the imitation learning rate $\alpha_i$:

\begin{equation}
	\mathcal{V}_{M1} = Q_{\theta}(s_t, a_t^d) + \alpha_i|Q_{\theta}(s_t, a_t^d)|
\end{equation}

It is important to note that we use the Q-functions of the agent to be trained and not the Q-functions of the demonstrator. In addition, the probability of taking the demonstrator action for the demonstrator $\pi_{\phi}^d(a_t^d|s_t)$ is used in the entropy term:

\begin{equation}
	\mathcal{H}_{M1} = \log\pi_{\phi}^d(a_t^d|s_t)
\end{equation}

When operating in mode 2, we increase the probability of taking the demonstrator action using the imitation learning rate:

\begin{equation}
	\mathcal{H}_{M2} = \log\pi_{\phi}^d(a_t^d|s_t) + \alpha_i|\log\pi_{\phi}^d(a_t^d|s_t)|
\end{equation}

The value term in mode 2 is the one in the classical objective.

Mode 3 combines both modes by using the entropy term of mode 2 and the value term of mode 1. Modes 4-6 are similar to modes 1-3, but the probability of taking the demonstrator action in the current trained policy $\pi_{\phi}(a_t^d|s_t)$ is used:

\begin{equation}
	\mathcal{H}_{M4} = \log\pi_{\phi}(a_t^d|s_t)
\end{equation}

\begin{equation}
	\mathcal{H}_{M5} = \log\pi_{\phi}(a_t^d|s_t) + \alpha_i|\log\pi_{\phi}(a_t^d|s_t)|
\end{equation}

Table~\ref{tab:social-policy-update-modes} summarizes the modes and their entropy and value terms.

{
\renewcommand{\baselinestretch}{0.9} 
\normalsize
\begin{table}[htb]
\center
\begin{tabular}{|l|l|l|}
\hline
    \textbf{Mode} & \textbf{Entropy term} & \textbf{Value term} \\
  \hline\hline
  1 & $\mathcal{H}_{M1}$ &$\mathcal{V}_{M1}$
  \\ \hline
  2 & $\mathcal{H}_{M2}$& $\mathcal{V}_{SAC}$ \\ \hline
  3 & $\mathcal{H}_{M2}$& $\mathcal{V}_{M1}$ \\ \hline 
  4 & $\mathcal{H}_{M4}$ & $\mathcal{V}_{M1}$ \\ \hline
  5 & $\mathcal{H}_{M5}$ & $\mathcal{V}_{SAC}$ \\ \hline
  6 & $\mathcal{H}_{M5}$ &  $\mathcal{V}_{M1}$ \\ \hline 
\end{tabular}
  \caption[Modes for Social Agent focusing on the policy update.]{Modes for Social Agent focusing on the policy update. Mode 1 increases the estimated Q-Value of the demonstrator action, mode 2 increases the probability of the demonstrator action and mode 3 combines both. Mode 4-6 are similar, but the probability of the demonstrator actions in the current trained policy is used instead of their probabilities in the demonstrator policy.}
  \label{tab:social-policy-update-modes}
\end{table}
}

We conducted experiments testing different modes and imitation learning rates with various demonstrators. An overview of the experiments is provided in Table~\ref{tab:social1-params}, while the achieved $fossi\_energy\_consumption$ KPI values are shown in Figure~\ref{fig:social1-results}. In the figure, points of the same color from the same demonstrator represent different imitation learning rates. However, since there is no clear trend indicating which values achieve the best performance, no distinction is made in the figure. The red dashed line represents the comparison value of the SAC baseline agents, while the gray dashed line shows deviations of 0.5\% compared to the baseline. We consider any deviations greater than 0.5\% to be significant.

{
\renewcommand{\baselinestretch}{0.9} 
\normalsize
\begin{table}[htb]
\center
\begin{tabular}{c | c | c | c}
Demonstrator                   & Mode & $\alpha_i$ & Deterministic policy actions \\ \hline
2 random                       & 1-6  & 0.2                      & Yes                          \\ 
\rule{0pt}{3ex}% EXTRA vertical height  
\multirow[t]{2}{*}{2 random}      & 1-3  & 0.01, 0.2, 1, 1.5        & No                           \\
                               & 4-6  & 0.01, 0.2                & No                           \\
\rule{0pt}{3ex}% EXTRA vertical height  
\multirow[t]{2}{*}{4 random}      & 1-3  & 0.01, 0.2, 1, 1.5        & No                           \\
                               & 4-6  & 0.01, 0.2                & No                           \\
\rule{0pt}{3ex}% EXTRA vertical height  
\multirow[t]{2}{*}{Pretrained B5} & 1-3  & 0.01, 0.2, 1, 1.5        & No                           \\
                               & 4-6  & 0.01, 0.2, 1             & No                           \\
\rule{0pt}{3ex}% EXTRA vertical height  
\multirow[t]{2}{*}{Pretrained B6} & 1-3  & 0.01, 0.2, 1, 1.5        & No                           \\
                               & 4-6  & 0.01, 0.2, 1             & No                          
\end{tabular}
  \caption{}
  \label{tab:social1-params}
\end{table}
}
\todo[inline]{early stopping}
As demonstrators, we used either the pre-trained buildings 5 and 6 presented in Section~\ref{sec:pretrained-demos} or either two or four random demonstrators, which are buildings from the training buildings and not pre-trained. To ensure comparability, we always used the same random demonstrators, which were Buildings 7 and 11 for two demonstrators and additionally Buildings 5 and 17 for four demonstrators.

In the first step, we sampled the demonstrator actions from the demonstrator policy instead of using the Gaussian distribution's learned mean. We tested the imitation learning rates of 0.01 and 0.2 and compared their results. We found that modes 4-6 demonstrated lower performance than the baseline, with slightly better results when using the pre-trained demonstrators than the random demonstrators. Modes 1-3 showed similar performance to the baseline for all demonstrators. Based on these results, we tried even higher imitation learning rates of 1 for modes 1-3 and 4-6 for pre-trained demonstrators and 1.5 for modes 1-3, but it did not lead to better results. Finally, for the two random demonstrators, with an imitation learning rate of 0.2 and deterministic policy actions, we also observed that modes 1-3 performed similarly to the baseline and modes 4-6 performed worse.

\begin{figure}[htb]
\center
     \includegraphics[width=\textwidth]{figures/social1_results.pdf}
  \caption{}
  \label{fig:social1-results}
\end{figure}


\section{Social Agent II}
Since the social policy update did not improve the performance of the models, we decided to focus on the Q-Value update in the next step. For this, we also first conducted classical Q-Value and policy updates as defined in Algorithm \ref{alg:sac-update}. Next, we performed a social Q-Value update on both Q-Value networks where we defined the target values as follows:
\begin{equation}
	J_Q^{social}(\theta)=\mathbb{E}_{s_t\sim\mathcal{D}}\left[\frac{1}{2}\left ( Q_{\theta}(s_t,a_t^d) - \left(Q_{\bar{\theta}}(s_t,a_t^d) + \alpha_i \left|Q_{\bar{\theta}}(s_t,a_t^d)\right| \right) \right )^2 \right],
\end{equation}

This follows the assumption of decision biasing, that actions taken by the demonstrator $a_t^d$ are good and should thus have a higher Q-Value as proposed by \cite{najar2020actions}. Afterwards, a soft update of the target networks parameters is performed, and, optionally, a standard policy update.

{
\renewcommand{\baselinestretch}{0.9} 
\normalsize
\begin{table}[htb]
\center
\begin{tabularx}{\textwidth}{c | Y | Y | Y}
Demonstrator                   & $\alpha_i$                                     & Deterministic policy actions & Shared observations \\ \hline
\rule{0pt}{3ex}% EXTRA vertical height  
\multirow[t]{3}{*}{2 random}      & $1e^{-4}$, $1e^{-3}$, 0.01, 0.03, 0.05, 0.1, 0.15, 0.2                & No                           & No                  \\
\rule{0pt}{2ex}% EXTRA vertical height  
                               & 0.03, 0.05, 0.1, 0.15, 0.2, 0.25                               & No                           & Yes                 \\
\rule{0pt}{2ex}% EXTRA vertical height  
                               & 0.1, 0.15, 0.2, 0.25                                           & Yes                          & Yes                 \\
\rule{0pt}{3ex}% EXTRA vertical height  
4 random                       & $1e^{-4}$, $1e^{-3}$, 0.01, 0.03, 0.05, 0.1, 0.15                     & No                           & No                  \\
\rule{0pt}{3ex}% EXTRA vertical height  
Pretrained B5                  & $1e^{-4}$, $1e^{-3}$, 0.01, 0.03, 0.05, 0.1, 0.15, 0.2, 0.4, 0.6, 0.8 & No                           & No                  \\
\rule{0pt}{3ex}% EXTRA vertical height  
\multirow[t]{2}{*}{Pretrained B6} & $1e^{-4}$, $1e^{-3}$, 0.01, 0.03, 0.05, 0.1, 0.15, 0.2, 0.4, 0.6, 0.8 & No                           & No                  \\
\rule{0pt}{2ex}% EXTRA vertical height  
                               & 0.1, 0.15, 0.2, 0.4, 0.6, 0.8                                  & Yes                          & No                 
\end{tabularx}
  \caption{}
  \label{tab:social2-params}
\end{table}
}

\begin{figure}[htb]
\center
     \includegraphics[width=\textwidth]{figures/social2_results.pdf}
  \caption{}
  \label{fig:social2-results}
\end{figure}

\section{Shifted loads}
Social Agent II with shifted loads





\cleardoublepage

%%
%%%%%%%%%%%%%%%%%%%%%%%%%%%%%%%%%%%%%%%%%%%%%%%%%%%%%%%%%%%%%%%%%%%%
% Diskussion und Ausblick
%%%%%%%%%%%%%%%%%%%%%%%%%%%%%%%%%%%%%%%%%%%%%%%%%%%%%%%%%%%%%%%%%%%%

\chapter{Discussion and Outlook}
  \label{Discussion}
  
\section{Evaluation on new Buildings}
In the preceding sections, we presented the theoretical foundations and development of our social approach to reducing fossil energy consumption. This chapter will evaluate our method on a new set of buildings. For the assessment, we selected the six buildings 1, 2, 4, 6, 9 and 14 (see Section~\ref{sec:building-data}), which we will refer to as evaluation buildings. These buildings were not included in the training set to avoid overlap. Using the same hyperparameters, reward function, and early stopping method, we first used the asocial SAC algorithm to train each building.

\begin{figure}[htb]
\center
     \includegraphics[width=\textwidth]{figures/eval_kpis.pdf}
  \caption{}
  \label{fig:eval-kpis}
\end{figure}

Figure~\ref{fig:eval-kpis} shows the KPI results of the RBC and SAC agents for both the training and evaluation buildings. The performance of the RBC agents is consistent across the building sets, with a slight improvement in the utilization of produced solar energy in the evaluation group. This indicates that when deploying RBC in evaluation buildings, the increase in the use of fossil fuels compared to without using the battery is similar to that of training buildings. 

On the other hand, the SAC agents of the evaluation buildings perform better than the baseline SAC agents in all KPIs except for the share of total renewable energy used. Notably, using battery storage in evaluation buildings facilitates an additional reduction of fossil energy consumption by approximately 1 \% compared to the training group.

\begin{figure}[htb]
\center
     \includegraphics[width=\textwidth]{figures/eval_results.pdf}
  \caption{}
  \label{fig:eval-results}
\end{figure}

To evaluate the effectiveness of our social method, we used a pre-trained demonstrator to train Social Agent II for the evaluation buildings. We used deterministic demonstrator actions and chose B11 and B6 as pre-trained demonstrators based on their correlation in energy consumption with the evaluation buildings. B11 has the highest median correlation without being part of the evaluation group, and B6 has the highest correlation within the group and performed the best in the training phase. We tested various imitation learning rates to determine the need for parameter tuning.

The mean value of the fossil energy consumption KPI was calculated for one experiment with and without additional policy update, as shown in Figure \ref{fig:eval-results}. Based on our analysis, we found that demonstrators B6 and B11 achieved the highest average saving of fossil energy at imitation learning rates of 0.2 and 0.3, respectively. We also discovered that the agents perform best when using demonstrator B6 with a rate of 0.25 and an additional policy update. Under these conditions, we can save about 1 \% more fossil energy than with the classic SAC agents. However, the savings achieved are lower compared to the training buildings.

  
\section{Final Discussion}
policy loss good indicator for final performance and stability of results

 Experiments mostly only one time --> more robust results if more often and than e.g. mean 
 
 MARLISA performance in paper unclear since compared to RBC, but values of RBC not given and in our case RBC worse than without battery

Social I:
Operates on policy update --> increasing value decreases the loss (not wanted), but increasing the probability more shifts the action to even more randomness. Also not exactly what we aimed. 


forwards perfect forward (but also for baseline)

pretrained Demonstrators trained on the same data of year (same weather, same fuelmix time series)

additional evaluation on new weather and consumption data of same buildings

pearson  correlation only linear
\todo[inline]{demonstrator policy update could be tried to improve more, e.g. second autotuned learning rate, other imitation learning rates, etc.}

\section{Outlook}

\subsection{Value Shaping}
Reward from Demonstrator mit einbeziehen ($-->$ Value Shaping)

In paper value function is updated, very similar to our social agent II. but since we use the absolute things there, its still frequency depending 

\subsection{Cluster Buildings}
Cluster by e.g. energy consumption or size of battery compared to consumption or PV etc etc and then use demonstrator per cluster
\cleardoublepage

%%
%%%%%%%%%%%%%%%%%%%%%%%%%%%%%%%%%%%%%%%%%%%%%%%%%%%%%%%%%%%%%%%%%%%%
% Zusammenfassung
%%%%%%%%%%%%%%%%%%%%%%%%%%%%%%%%%%%%%%%%%%%%%%%%%%%%%%%%%%%%%%%%%%%%

\chapter{Conclusion}
  \label{conclusion}
\todo[inline]{TODO}

\clearpage

\cleardoublepage

%%%%%%%%%%%%%%%%%%%%%%%%%%%%%%%%%%%%%%%%%%%%%%%%%%%%%%%%%%%%%%%%%%%%%%%%%%%%%
%%% Appendix
%%%%%%%%%%%%%%%%%%%%%%%%%%%%%%%%%%%%%%%%%%%%%%%%%%%%%%%%%%%%%%%%%%%%%%%%%%%%%
%%%%%%%%%%%%%%%%%%%%%%%%%%%%%%%%%%%%%%%%%%%%%%%%%%%%%%%%%%%%%%%%%%%%
% Anhang
%%%%%%%%%%%%%%%%%%%%%%%%%%%%%%%%%%%%%%%%%%%%%%%%%%%%%%%%%%%%%%%%%%%%

\appendix

%\setcounter{secnumdepth}{-1}
\chapter{Appendix}\label{chap:App}
\section{Tables}\label{chap:App}
\begin{table}[H]
\begin{tabularx}{\linewidth}{lXc}
Attribute name & Description & Value \\ \hline
Capacity & Maximum amount of energy the storage device can store & 6.4 kWh \\
Efficiency & Technical efficiency & 90 \% \\
Capacity loss coefficient & Storage capacity lost in each charge and discharge cycle (as a fraction of the total capacity) & $1e^{-5}$ \\
Loss coefficient & Standby hourly losses & 0 \\
Nominal power & Maximum amount of electric power that the battery can use to charge or discharge & 5 \textcolor{red}{kW}
\end{tabularx}
\caption[Attributes of the batteries of each building]{Attributes of the batteries of each building. Description of the attributes obtained from \textcolor{red}{TODO source}}
\label{app:battery-attributes}
\end{table}

\begin{table}[H]
\center
\begin{tabular}{c|c}
Building Id & Energy Consumption Median 
%& Energy Consumption without storage Median & Energy Consumption without storage and PV Median 
\\  \hline
1  & 0.54 %& 0.58 & 0.81 
\\
2  & 0.45 %& 0.49 & 0.74 
\\
3  & 0.25 %& 0.28 & 0.56 
\\
4 & 0.52 %& 0.54 & 0.94  
\\
5 & 0.29 %& 0.29 & 0.74 
\\
6 & 0.5 % & 0.56 & 1.08 
\\
7 & 0.21 %& 0.21 & 0.33 
\\
8 & 0.31 %& 0.32 & 0.68 
\\
9  & 0.36 %& 0.38 & 0.46 
\\
10 & 0.56 %& 0.65 & 0.98
\\
11 & 0.66 %& 0.71 & 1.19 
\\
12 & 0.0 %& 0.0 & 0.0 
\\
13 & 0.43 %& 0.48 & 0.97 
\\
14 & 0.49 %& 0.48 & 0.59 
\\
15 & 0.12 %& 0.0 & 0.0 
\\
16 & 0.55 %& 0.72 & 1.2 
\\
17 & 0.75% & 0.82 & 1.24 \\
\end{tabular}
\caption{\textcolor{red}{Zweite Appendix-Tabelle}}
\label{tab:building-medians}
\end{table}

\begin{table}[H]
\begin{tabularx}{\linewidth}{l|X}
Name & Description \\ \hline
month & 1 (January) through 12 (December)\\
day\_type & type of day as provided by EnergyPlus (from 1 to 8). 1 (Sunday), 2 (Monday), ..., 7 (Saturday), 8 (Holiday)\\
hour & hour of day (from 1 to 24)\\
 t\_out & outdoor temperature in Celcius degrees.\\
 t\_out\_pred\_6h & outdoor temperature predicted 6h ahead \\
 t\_out\_pred\_12h & outdoor temperature predicted 12h ahead \\
 t\_out\_pred\_24h & outdoor temperature predicted 24h ahead \\
 rh\_out & outdoor relative humidity in \%.\\
 rh\_out\_pred\_6h & outdoor relative humidity predicted 6h ahead \\
 rh\_out\_pred\_12h & outdoor relative humidity predicted 12h ahead \\
 rh\_out\_pred\_24h & outdoor relative humidity predicted 24h ahead \\
 diffuse\_solar\_rad & diffuse solar radiation in $W/m^2$.\\
 diffuse\_solar\_rad\_pred\_6h & diffuse solar radiation predicted 6h ahead \\
 diffuse\_solar\_rad\_pred\_12h & diffuse solar radiation predicted 12h ahead \\
 diffuse\_solar\_rad\_pred\_24h & diffuse solar radiation predicted 24h ahead \\
 direct\_solar\_rad & direct solar radiation in $W/m^2$. \\
 direct\_solar\_rad\_pred\_6h & direct solar radiation predicted 6h ahead \\ 
 direct\_solar\_rad\_pred\_12h & direct solar radiation predicted 12h ahead \\ 
 direct\_solar\_rad\_pred\_24h & direct solar radiation predicted 24h ahead \\ 
 wind\_speed & wind speed in $m/s$ \\
 wind\_speed\_pred\_6h & wind speed predicted 6h ahead \\
 wind\_speed\_pred\_12h & wind speed predicted 12h ahead \\
 wind\_speed\_pred\_24h & wind speed predicted 24h ahead \\
 non\_shiftable\_load & electricity currently consumed by electrical appliances in kWh.\\ 
 solar\_gen & electricity currently being generated by photovoltaic panels in kWh.\\ 
 electrical\_storage\_soc & SOC of the electrical storage from 0 (no energy stored) to 1 (at full capacity). \\
 net\_electricity\_consumption & net electricity consumption of the building (including all energy systems) in the current time step. \\
 electricity\_pricing & electricity rate in $\$/kWh$ \\
 electricity\_pricing\_pred\_6h & electricity rate predicted 6 hours ahead. \\
 electricity\_pricing\_pred\_12h & electricity rate predicted 12 hours ahead. \\
 electricity\_pricing\_pred\_24h & electricity rate predicted 24 hours ahead. \\
 renewable\_energy\_share & share of renewable energy in the fuel mix of the grid.
\end{tabularx}
\caption{Available state variables of the baseline SAC Agent}
\label{app:observation-space-sac}
\end{table}

\section{Figures}\label{sec:app-figures}
\begin{figure}[H]
    \center
     \includegraphics[width=\textwidth]{figures/building_daily_means.pdf}
  \caption{Daily means of the non-shiftable load and the solar generation of the buildings in kWh.}
  \label{app:building-daily-mean}
\end{figure}

\begin{figure}[H]
    \center
     \includegraphics[width=\textwidth]{figures/weather_daily_means.pdf}
  \caption{}
  \label{app:weather-daily-mean}
\end{figure}

\begin{figure}[H]
    \center
     \includegraphics[width=\textwidth]{figures/prb_ddpg_kpis.pdf}
  \caption{}
  \label{app:prb-ddpg-kpis}
\end{figure}


\section{Algorithms}\label{sec:app-algos}
\begin{algorithm}
\caption{\textcolor{red}{Social Agent I}}
\label{app:sac-social-1}
\small\textbf{Input:}   $\theta_1,\theta_2,\phi$ \Comment{\footnotesize Initial parameters}
\begin{algorithmic}[1]
\small \State $\bar{\theta}_1 \gets \theta_1, \bar{\theta}_2 \gets \theta_2$ \Comment{\footnotesize Initialize target network weights}
\small \State $\mathcal{D} \gets \emptyset$ \Comment{\footnotesize Initialize an empty replay buffer}
\small \For{each iteration}
	\small \For{each environment step}
		\small \State $\mathbf{a}_t \sim \pi_{\phi}(\mathbf{a}_t|\mathbf{s}_t)$ \Comment{\footnotesize Sample action from the policy}
		\small \State $\mathbf{s}_{t+1} \sim p(\mathbf{s}_{t+1}|\mathbf{s}_t,\mathbf{a}_t)$ \Comment{\footnotesize Sample next state from the environment}
		\small \State $\mathcal{D} \gets \mathcal{D} \cup \{(\mathbf{s}_t,\mathbf{a}_t,r(\mathbf{s}_t,\mathbf{a}_t),\mathbf{s}_{t+1})\}$ \Comment{\footnotesize Store the transition in the replay buffer}
	\small \EndFor
	\small \For{each gradient step}
		\small \State $\theta_i \gets \theta_i - \lambda_Q \hat{\triangledown}_{\theta_i}J_Q(\theta_i)$ for $i \in \{1,2\}$ \Comment{\footnotesize Update Q-function parameters}
		\small \State $\phi \gets \phi - \lambda _{\pi} \hat{\triangledown}_{\phi}J_{\pi}(\phi)$ \Comment{\footnotesize Update policy weights}
		\small \State $\alpha \gets \alpha - \lambda  \hat{\triangledown}_{\alpha}J(\alpha)$ \Comment{\footnotesize Adjust temperature}
		\small \State $\bar{\theta}_i \gets \tau\theta_i + (1-\tau)\bar{\theta}_i$ for $i \in \{1,2\}$ \Comment{\footnotesize Update target network weights}
		\small \For{each demonstrator $d$}
			\small \State $\mathbf{a}_t^d \sim \pi_{\phi}^d(\mathbf{a}_t^d|\mathbf{s}_t)$
			\small \State $\phi \gets \phi - \lambda _{\pi} \hat{\triangledown}_{\phi}J_{\pi}^{social}(\phi)$ \Comment{\footnotesize Social policy update}
		\small \EndFor
	\small \EndFor
\small \EndFor
\end{algorithmic}
\small \textbf{Output:}   $\theta_1,\theta_2,\phi$ \Comment{Optimized parameters}
\end{algorithm}

\begin{algorithm}
\caption{\textcolor{red}{Social Agent II}}
\label{app:sac-social-2}
\small\textbf{Input:}   $\theta_1,\theta_2,\phi$ \Comment{\footnotesize Initial parameters}
\begin{algorithmic}[1]
\small \State $\bar{\theta}_1 \gets \theta_1, \bar{\theta}_2 \gets \theta_2$ \Comment{\footnotesize Initialize target network weights}
\small \State $\mathcal{D} \gets \emptyset$ \Comment{\footnotesize Initialize an empty replay buffer}
\small \For{each iteration}
	\small \For{each environment step}
		\small \State $\mathbf{a}_t \sim \pi_{\phi}(\mathbf{a}_t|\mathbf{s}_t)$ \Comment{\footnotesize Sample action from the policy}
		\small \State $\mathbf{s}_{t+1} \sim p(\mathbf{s}_{t+1}|\mathbf{s}_t,\mathbf{a}_t)$ \Comment{\footnotesize Sample next state from the environment}
		\small \State $\mathcal{D} \gets \mathcal{D} \cup \{(\mathbf{s}_t,\mathbf{a}_t,r(\mathbf{s}_t,\mathbf{a}_t),\mathbf{s}_{t+1})\}$ \Comment{\footnotesize Store the transition in the replay buffer}
	\small \EndFor
	\small \For{each gradient step}
		\small \State $\theta_i \gets \theta_i - \lambda_Q \hat{\triangledown}_{\theta_i}J_Q(\theta_i)$ for $i \in \{1,2\}$ \Comment{\footnotesize Update Q-function parameters}
		\small \State $\phi \gets \phi - \lambda _{\pi} \hat{\triangledown}_{\phi}J_{\pi}(\phi)$ \Comment{\footnotesize Update policy weights}
		\small \State $\alpha \gets \alpha - \lambda  \hat{\triangledown}_{\alpha}J(\alpha)$ \Comment{\footnotesize Adjust temperature}
		\small \State $\bar{\theta}_i \gets \tau\theta_i + (1-\tau)\bar{\theta}_i$ for $i \in \{1,2\}$ \Comment{\footnotesize Update target network weights}
		\small \For{each demonstrator $d$}
			\small \State $\theta_i \gets \theta_i - \lambda_Q \hat{\triangledown}_{\theta_i}J_Q^{social}(\theta_i)$ for $i \in \{1,2\}$ \Comment{\footnotesize Social Q-function update}
			\small \State $\bar{\theta}_i \gets \tau\theta_i + (1-\tau)\bar{\theta}_i$ for $i \in \{1,2\}$ \Comment{\footnotesize Update target network weights}
			\small \If{extra policy update}
				\small \State $\phi \gets \phi - \lambda _{\pi} \hat{\triangledown}_{\phi}J_{\pi}(\phi)$ \Comment{\footnotesize Update policy weights}
			\small \EndIf
		\small \EndFor
	\small \EndFor
\small \EndFor
\end{algorithmic}
\small \textbf{Output:}   $\theta_1,\theta_2,\phi$ \Comment{Optimized parameters}
\end{algorithm}

%\end{appendices)

\cleardoublepage

%%%%%%%%%%%%%%%%%%%%%%%%%%%%%%%%%%%%%%%%%%%%%%%%%%%%%%%%%%%%%%%%%%%%%%%%%%%%%
%%% Bibliographie
%%%%%%%%%%%%%%%%%%%%%%%%%%%%%%%%%%%%%%%%%%%%%%%%%%%%%%%%%%%%%%%%%%%%%%%%%%%%%

\addcontentsline{toc}{chapter}{Bibliography}

\bibliographystyle{alpha}
\bibliography{thesislit}
%% Obige Anweisung legt fest, dass BibTeX-Datei `mylit.bib' verwendet
%% wird. Hier koennen mehrere Dateinamen mit Kommata getrennt aufgelistet
%% werden.

\cleardoublepage
%%%%%%%%%%%%%%%%%%%%%%%%%%%%%%%%%%%%%%%%%%%%%%%%%%%%%%%%%%%%%%%%%%%%%%%%%%%%%
%%% Erklaerung
%%%%%%%%%%%%%%%%%%%%%%%%%%%%%%%%%%%%%%%%%%%%%%%%%%%%%%%%%%%%%%%%%%%%%%%%%%%%%
\thispagestyle{empty}
\section*{Selbst\"andigkeitserkl\"arung}

Hiermit versichere ich, dass ich die vorliegende Masterarbeit 
selbst\"andig und nur mit den angegebenen Hilfsmitteln angefertigt habe und dass alle Stellen, die dem Wortlaut oder dem 
Sinne nach anderen Werken entnommen sind, durch Angaben von Quellen als 
Entlehnung kenntlich gemacht worden sind. 
Diese Masterarbeit wurde in gleicher oder \"ahnlicher Form in keinem anderen 
Studiengang als Pr\"ufungsleistung vorgelegt. 

\vskip 3cm

Ort, Datum	\hfill Unterschrift \hfill 
%%%%%%%%%%%%%%%%%%%%%%%%%%%%%%%%%%%%%%%%%%%%%%%%%%%%%%%%%%%%%%%%%%%%%%%%%%%%%
%%% Ende
%%%%%%%%%%%%%%%%%%%%%%%%%%%%%%%%%%%%%%%%%%%%%%%%%%%%%%%%%%%%%%%%%%%%%%%%%%%%%

\end{document}

