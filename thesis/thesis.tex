%%%%%%%%%%%%%%%%%%%%%%%%%%%%%%%%%%%%%%%%%%%%%%%%%%%%%%%%%%%%%%%%%%%%%%%%%%%%%
%%% LaTeX-Rahmen fuer das Erstellen von Masterarbeiten
%%%%%%%%%%%%%%%%%%%%%%%%%%%%%%%%%%%%%%%%%%%%%%%%%%%%%%%%%%%%%%%%%%%%%%%%%%%%%

%%%%%%%%%%%%%%%%%%%%%%%%%%%%%%%%%%%%%%%%%%%%%%%%%%%%%%%%%%%%%%%%%%%%%%%%%%%%%
%%% allgemeine Einstellungen
%%%%%%%%%%%%%%%%%%%%%%%%%%%%%%%%%%%%%%%%%%%%%%%%%%%%%%%%%%%%%%%%%%%%%%%%%%%%%

\documentclass[twoside,12pt,a4paper]{report}
%\usepackage{reportpage}
\usepackage{epsf}
\usepackage{graphics, graphicx}
\usepackage{latexsym}
\usepackage[margin=10pt,font=small,labelfont=bf]{caption}
\usepackage[utf8]{inputenc}
\usepackage[toc,page]{appendix}


% Own packages
\usepackage[colorinlistoftodos]{todonotes}
\usepackage{algorithm}
\usepackage{algpseudocode}
\usepackage{amssymb}
\usepackage{tabularx}
\usepackage{url}
\usepackage{amsmath}

\newcommand\numberthis{\addtocounter{equation}{1}\tag{\theequation}}

\textwidth 14cm
\textheight 22cm
\topmargin 0.0cm
\evensidemargin 1cm
\oddsidemargin 1cm
%\footskip 2cm
\parskip0.5explus0.1exminus0.1ex

% Kann von Student auch nach pers\"onlichem Geschmack ver\"andert werden.
\pagestyle{headings}

\sloppy

\begin{document}

%%%%%%%%%%%%%%%%%%%%%%%%%%%%%%%%%%%%%%%%%%%%%%%%%%%%%%%%%%%%%%%%%%%%%%%%%%%%
%%% hier steht die neue Titelseite 
%%%%%%%%%%%%%%%%%%%%%%%%%%%%%%%%%%%%%%%%%%%%%%%%%%%%%%%%%%%%%%%%%%%%%%%%%%%%
 
\begin{titlepage}
 \begin{center}
  {\LARGE Eberhard Karls Universit\"at T\"ubingen}\\
  {\large Mathematisch-Naturwissenschaftliche Fakult\"at \\
Wilhelm-Schickard-Institut f\"ur Informatik\\[4cm]}
  {\huge Master Thesis Computer Science\\[2cm]}
  {\Large\bf  Incorporating Social Learning into Multi-Agent Reinforcement Learning
to \\Lower Carbon Emissions in Energy Systems\\[1.5cm]}
 {\large Kristina Lietz}\\[0.5cm]
01.12.2023\\[4cm]
{\small\bf Reviewers}\\[0.5cm]
  \parbox{7cm}{\begin{center}{\large Dr. Nicole Ludwig}\\
   (Bioinformatik)\\
  {\footnotesize Wilhelm-Schickard-Institut f\"ur Informatik\\
	Universit\"at T\"ubingen}\end{center}}\hfill\parbox{7cm}{\begin{center}
  {\large Prof. Setareh Maghsudit}\\
  (Biologie/Medizin)\\
  {\footnotesize Medizinische Fakult\"at\\
	Universit\"at T\"ubingen}\end{center}
 }
  \end{center}
\end{titlepage}

%%%%%%%%%%%%%%%%%%%%%%%%%%%%%%%%%%%%%%%%%%%%%%%%%%%%%%%%%%%%%%%%%%%%%%%%%%%%
%%% Titelr"uckseite: Bibliographische Angaben
%%%%%%%%%%%%%%%%%%%%%%%%%%%%%%%%%%%%%%%%%%%%%%%%%%%%%%%%%%%%%%%%%%%%%%%%%%%%

\thispagestyle{empty}
\vspace*{\fill}
\begin{minipage}{11.2cm}
\textbf{Lietz, Kristina:}\\
\emph{Incorporating Social Learning into Multi-Agent Reinforcement Learning
to Lower Carbon Emissions in Energy Systems}\\ Master Thesis Computer Science\\
Eberhard Karls Universit\"at T\"ubingen\\
Thesis period: 01.06.2023~-~01.12.2023
\end{minipage}
\newpage

%%%%%%%%%%%%%%%%%%%%%%%%%%%%%%%%%%%%%%%%%%%%%%%%%%%%%%%%%%%%%%%%%%%%%%%%%%%%

\pagenumbering{roman}
\setcounter{page}{1}

%%%%%%%%%%%%%%%%%%%%%%%%%%%%%%%%%%%%%%%%%%%%%%%%%%%%%%%%%%%%%%%%%%%%%%%%%%%%
%%% Seite I: Zusammenfassug, Danksagung
%%%%%%%%%%%%%%%%%%%%%%%%%%%%%%%%%%%%%%%%%%%%%%%%%%%%%%%%%%%%%%%%%%%%%%%%%%%%


\section*{Abstract}

\todo[inline]{Write here your abstract.}

\newpage
\section*{Zusammenfassung}

\todo[inline]{Bei einer englischen Masterarbeit muss zus\"atzlich eine deutsche Zusammenfassung verfasst werden.}

\newpage
\section*{Acknowledgements}

\todo[inline]{Write here your acknowledgements.}

\cleardoublepage

%%%%%%%%%%%%%%%%%%%%%%%%%%%%%%%%%%%%%%%%%%%%%%%%%%%%%%%%%%%%%%%%%%%%%%%%%%%%%
%%% Table of Contents
%%%%%%%%%%%%%%%%%%%%%%%%%%%%%%%%%%%%%%%%%%%%%%%%%%%%%%%%%%%%%%%%%%%%%%%%%%%%%

\renewcommand{\baselinestretch}{1.3}
\small\normalsize

\tableofcontents

\renewcommand{\baselinestretch}{1}
\small\normalsize

\cleardoublepage

%%%%%%%%%%%%%%%%%%%%%%%%%%%%%%%%%%%%%%%%%%%%%%%%%%%%%%%%%%%%%%%%%%%%%%%%%%%%%
%%% List of Figures
%%%%%%%%%%%%%%%%%%%%%%%%%%%%%%%%%%%%%%%%%%%%%%%%%%%%%%%%%%%%%%%%%%%%%%%%%%%%%

\renewcommand{\baselinestretch}{1.3}
\small\normalsize

\addcontentsline{toc}{chapter}{List of Figures}
\listoffigures

\renewcommand{\baselinestretch}{1}
\small\normalsize

\cleardoublepage

%%%%%%%%%%%%%%%%%%%%%%%%%%%%%%%%%%%%%%%%%%%%%%%%%%%%%%%%%%%%%%%%%%%%%%%%%%%%%
%%% List of tables
%%%%%%%%%%%%%%%%%%%%%%%%%%%%%%%%%%%%%%%%%%%%%%%%%%%%%%%%%%%%%%%%%%%%%%%%%%%%%

\renewcommand{\baselinestretch}{1.3}
\small\normalsize

\addcontentsline{toc}{chapter}{List of Tables}
\listoftables

\renewcommand{\baselinestretch}{1}
\small\normalsize

\cleardoublepage

%%%%%%%%%%%%%%%%%%%%%%%%%%%%%%%%%%%%%%%%%%%%%%%%%%%%%%%%%%%%%%%%%%%%%%%%%%%%%
%%% List of abbreviations
%%%%%%%%%%%%%%%%%%%%%%%%%%%%%%%%%%%%%%%%%%%%%%%%%%%%%%%%%%%%%%%%%%%%%%%%%%%%%

% can be removed
\addcontentsline{toc}{chapter}{List of Abbreviations}
\chapter*{List of Abbreviations\markboth{LIST OF ABBREVIATIONS}{LIST OF ABBREVIATIONS}}
\todo[inline]{TODO}
\begin{tabbing}
\textbf{FACTOTUM}\hspace{1cm}\=Schrott\kill
\textbf{KL divergence}\>Kullback-Leibler divergence \\
\textbf{KPI}\>Key performance indicator \\
\textbf{kWh}\>Kilowatt hour \\
\textbf{MDP}\>Markov decision process \\
\textbf{NY}\> New York \\
\textbf{PV}\> Photovoltaic \\
\textbf{RL}\>Reinforcement learning \\
\textbf{SAC}\>Soft actor-critic \\
\textbf{SOC}\> State of the charge \\
\textbf{TD error}\> Temporal Difference error \\
\textbf{...} \> ...\\
\end{tabbing}

\cleardoublepage

%%%%%%%%%%%%%%%%%%%%%%%%%%%%%%%%%%%%%%%%%%%%%%%%%%%%%%%%%%%%%%%%%%%%%%%%%%%%%
%%% Der Haupttext, ab hier mit arabischer Numerierung
%%% Mit \input{dateiname} werden die Datei `dateiname' eingebunden
%%%%%%%%%%%%%%%%%%%%%%%%%%%%%%%%%%%%%%%%%%%%%%%%%%%%%%%%%%%%%%%%%%%%%%%%%%%%%

\pagenumbering{arabic}
\setcounter{page}{1}

%% Introduction
%%%%%%%%%%%%%%%%%%%%%%%%%%%%%%%%%%%%%%%%%%%%%%%%%%%%%%%%%%%%%%%%%%%%
% Einleitung
%%%%%%%%%%%%%%%%%%%%%%%%%%%%%%%%%%%%%%%%%%%%%%%%%%%%%%%%%%%%%%%%%%%%

\chapter{Introduction}\label{Introduction}

\todo[inline]{introduction, background (additional chapters), structure of thesis}


\begin{figure}[htb]
     \centerline{\epsffile{figures/chordal.eps}}
  \caption{Chordale Graphen}
  \label{fig2.1}
\end{figure}

Abbildung~\ref{fig2.1} zeigt ...

{
\renewcommand{\baselinestretch}{0.9} 
\normalsize
\begin{table}[htb]
\begin{tabular}{|p{2.7cm}||l|c|r|}
\hline
    \textbf{Spalte 1} 
  & \textbf{Spalte 2} 
  & \textbf{Spalte 3} 
  & \textbf{Spalte 4} \\
  \hline\hline
  xxx1111
  & xxxxxxx2222222
  & xxxxxx333333 
  & xxxxxxxxxx444444 \\
  \hline
    ...
  & ...
  & ...
  & ...\\
  \hline
\end{tabular}
  \caption[Beispieltabelle mit einer langen Legende]{Beispieltabelle mit einer langen Legende, damit man sieht, dass in der Legende der Zeilenabstand verringert wurde. Ausserdem soll auch der Font etwas kleiner gew\"ahlt werden. So sieht die ganze Umgebung kompakter aus.}
  \label{tabelle-1}
\end{table}
}
Referenzen: \cite{SaaSchTue97,TueConSaa96ismis,SchTueSaa98preprint}

\cleardoublepage

%% 
\chapter{Background}
\label{Background}
\todo[inline]{Begriffserklärung: renewable energy (wind, solar, bla)}
This chapter provides the necessary background knowledge to comprehend our approach to the presented problem. First, we introduce the fundamental concepts of reinforcement learning (RL), often modeled as finite Markov decision processes (MDP). However, considering the limitations of this traditional framework in handling continuous state and action spaces, we also delve into the soft actor-critic method as a promising solution.
\todo[inline]{TODO}

\input{Background}
\cleardoublepage

%% 
%%%%%%%%%%%%%%%%%%%%%%%%%%%%%%%%%%%%%%%%%%%%%%%%%%%%%%%%%%%%%%%%%%%%
% Grundlagen
%%%%%%%%%%%%%%%%%%%%%%%%%%%%%%%%%%%%%%%%%%%%%%%%%%%%%%%%%%%%%%%%%%%%

\subsection{SAC using demonstrator transitions}
\todo[inline]{Font in tabellen beschreibungen verkleinern}
As mentioned in Section \ref{sec:background-social-learning}, a widely used method to improve learning is sampling demonstrator transitions and storing them in a Prioritized Replay Buffer. These kind of buffer add priorities based on the Temporal difference (TD) error to the transitions: Transistion with a higher TD error have an higher priority, assuming that these are more difficult to learn and hence should be seen more often during training. Thus, transitions with a higher priority are sampled with a higher probability. The priorities are updated during the training process \cite{schaul2015prioritized}.

\todo[inline]{Present the buildings used in training and the demonstrator building (and why this)}
For implementing the SAC agent using the demonstrator transitions, we first trained one building with 

\subsection{\textcolor{red}{Social Agent I}}
\textcolor{red}{Policy update}
First: normal policy update

Then: policy update using demonstrator actions

Classical loss $l$ to minimize is is 
\begin{align*}
	l &= \mathbb{E}_{s_t \sim \mathcal{D}, \epsilon_t \sim \mathcal{N}}[\alpha \log\pi_{\phi}(f_{\phi}(\epsilon_t;s_t)|s_t)-Q_{\theta}(s_t, f_{\phi}(\epsilon_t;s_t))] \\ 
	&=  \mathbb{E}_{s_t \sim \mathcal{D}, \epsilon_t \sim \mathcal{N}}[\alpha \mathcal{H}_{SAC}-\mathcal{V}_{SAC}] \numberthis \label{eqn:policy-loss}
\end{align*}
with the entropy term that aims to maximize randomness $\mathcal{H}_{SAC}$ and the value term that aims to maximize the estimated Q-Value of the action $\mathcal{V}_{SAC}$. For incorporating demonstrator actions similar to decision biasing described in chapter \ref{sec:decision-biasing}, so increase the probability of observed demonstrator actions or increase the estimated Q-Value of these, we modify the terms as follows: In mode 1, we use the actions sampled from the demonstrator $f^d_{\Phi}(\epsilon_t;s_t)$ in the value term and increase the estimated Q-Value by adding a fraction of the absolute value of it using the imitation learning rate $\alpha_i$:
\begin{equation}
	\mathcal{V}_{M1} =  Q_{\theta}(s_t, f_{\phi}^d(\epsilon_t;s_t)) + \alpha_i|Q_{\theta}(s_t, f_{\phi}^d(\epsilon_t;s_t))|
\end{equation}
Also, the probability of taking the demonstrator action for the demonstrator is used in the entropy term:
\begin{equation}
	\mathcal{H}_{M1} =  \log\pi_{\phi}^d(f_{\phi}^d(\epsilon_t;s_t)|s_t)
\end{equation}
In mode 2, the probability of taking the demonstrator action is increased by the learning rate:
\begin{equation}
	\mathcal{H}_{M2} = \log\pi_{\phi}^d(f_{\phi}^d(\epsilon_t;s_t)|s_t) + \alpha_i|\log\pi_{\phi}^d(f_{\phi}^d(\epsilon_t;s_t)|s_t)|
\end{equation}
Mode 3 combines mode 1 and mode 2. Mode 4-6 are similar to the modes 1-3, but the probability of taking the demonstrator action in the current trained policy is used:
\begin{equation}
	\mathcal{H}_{M4} =  \log\pi_{\phi}(f_{\phi}^d(\epsilon_t;s_t)|s_t)
\end{equation}
\begin{equation}
	\mathcal{H}_{M5} = \log\pi_{\phi}(f_{\phi}^d(\epsilon_t;s_t)|s_t) + \alpha_i|\log\pi_{\phi}(f_{\phi}^d(\epsilon_t;s_t)|s_t)|
\end{equation}
All modes calculate the loss as defined in Equation \ref{eqn:policy-loss} using the defined entropy and value term. Table \ref{tab:social-policy-update-modes} summarizes the different modes and their entropy and value terms.

{
\renewcommand{\baselinestretch}{0.9} 
\normalsize
\begin{table}[htb]
\center
\begin{tabular}{|l|l|l|}
\hline
    \textbf{Mode} & \textbf{Entropy term} & \textbf{Value term} \\
  \hline\hline
  1 & $\mathcal{H}_{M1}$ &$\mathcal{V}_{M1}$
  \\ \hline
  2 & $\mathcal{H}_{M2}$& $\mathcal{V}_{SAC}$ \\ \hline
  3 & $\mathcal{H}_{M2}$& $\mathcal{V}_{M1}$ \\ \hline 
  4 & $\mathcal{H}_{M4}$ & $\mathcal{V}_{M1}$ \\ \hline
  5 & $\mathcal{H}_{M5}$ & $\mathcal{V}_{SAC}$ \\ \hline
  6 & $\mathcal{H}_{M5}$ &  $\mathcal{V}_{M1}$ \\ \hline 
\end{tabular}
  \caption[Modes for Social Agent focusing on the policy update]{Modes for Social Agent focusing on the policy update. Mode 1 increases the estimated Q-Value of the demonstrator action, mode 2 increases the probability of the demonstrator action and mode 3 combines both. Mode 4-6 are similar, but the probability of the demonstrator actions in the current trained policy is used instead of their probabilities in the demonstrator policy.}
  \label{tab:social-policy-update-modes}
\end{table}
}


\subsection{\textcolor{red}{Social Agent II}}
Since the social policy update did not improve the performance of the models, we decided to focus on the Q-Value update in the next step. For this, we also first conducted classical Q-Value and policy updates as defined in Algorithm \ref{alg:sac-update}. Next, we performed a social Q-Value update on both Q-Value networks where we defined the target values as follows:
\begin{equation}
	Q_{\bar{\theta}}(a_t^d,s_t) + \alpha_i|Q_{\bar{\theta}}(a_t^d,s_t)| 
\end{equation}
This follows the assumption of decision biasing, that actions taken by the demonstrator $a_t^d$ are good and should thus have a higher Q-Value as proposed by \cite{najar2020actions}. Afterwards, a soft update of the target networks parameters is performed, and, optionally, a standard policy update.





\cleardoublepage

%%
%%%%%%%%%%%%%%%%%%%%%%%%%%%%%%%%%%%%%%%%%%%%%%%%%%%%%%%%%%%%%%%%%%%%
% Diskussion und Ausblick
%%%%%%%%%%%%%%%%%%%%%%%%%%%%%%%%%%%%%%%%%%%%%%%%%%%%%%%%%%%%%%%%%%%%

\chapter{Results}
  \label{results}
\todo[inline]{TODO}

\clearpage

Used PV of available\\
$possible\_battery\_input_{b,t} = \min(\min(C^b_t-SOC^b_t, 0), max\_input\_power_{b, \textcolor{red}{t=last}})$\\
\\
$could\_used\_without\_pv_{b,t} = possible\_battery\_input_{b,t} + net\_electricity\_consumption\_without\_storage\_and\_pv_{b,t} $ \\
\\
$could\_used\_with\_pv_{b,t} = \max(possible\_battery\_input_{b,t} + net\_electricity\_consumption\_without\_storage\_and\_pv_{b,t} - solar\_generation_{b,t}*-1, 0)$ \\
\\
$pv\_could\_have\_used_{b,t} = \min(solar\_generation_{b,t}*-1, could\_used\_without\_pv_{b,t})$ \\
\\
$pv\_used_{b,t} = E_{used_{pv_{b,t}}} = \max(\min(e_{net}^b - e_{pv}^b, - e_{pv}^b), 0) $ \\
\\
$pv\_used\_of\_available_{b,t} = used_{b,t}/pv\_could\_have\_used_{b,t}$\\
\\
$grid\_could\_have\_used_{t} = \min(E_{r,grid_{t}}, \sum_{b\in Buildings}could\_used\_with\_pv_{b,t})$ \\
\\
$grid\_used_{t} = E_{used_{r,grid_{t}}} = \min(E_{net_{pos}}, E_{r, grid})$ \\
\\
$grid\_used\_of\_available_{t} = grid\_used_{b,t} / grid\_could\_have\_used_{b,t}$ \\
\\
$total\_could\_have\_used_{t} = \min(E_{r,grid_{t}} + \sum_{b\in Buildings} \min(solar\_generation_{b,t}*-1, could\_used\_without\_pv_{b,t}), \sum_{b\in Buildings}could\_used\_with\_pv_{b,t}))$ \\
\\
$total\_used_{t} = E_{used_{r}} = E_{used_{r,grid}}  + E_{used_{pv}}$ \\
\\
$total\_used\_of\_available_{t} = total\_used_{t} / total\_could\_have\_used_{t}$


\cleardoublepage

%%
%%%%%%%%%%%%%%%%%%%%%%%%%%%%%%%%%%%%%%%%%%%%%%%%%%%%%%%%%%%%%%%%%%%%
% Diskussion und Ausblick
%%%%%%%%%%%%%%%%%%%%%%%%%%%%%%%%%%%%%%%%%%%%%%%%%%%%%%%%%%%%%%%%%%%%

\chapter{Discussion and Outlook}
  \label{Discussion}
  
\section{Evaluation on new Buildings}
In the preceding sections, we presented the theoretical foundations and development of our social approach to reducing fossil energy consumption. This chapter will evaluate our method on a new set of buildings. For the assessment, we selected the six buildings 1, 2, 4, 6, 9 and 14 (see Section~\ref{sec:building-data}), which we will refer to as evaluation buildings. These buildings were not included in the training set to avoid overlap. Using the same hyperparameters, reward function, and early stopping method, we first used the asocial SAC algorithm to train each building.

\begin{figure}[htb]
\center
     \includegraphics[width=\textwidth]{figures/eval_kpis.pdf}
  \caption{}
  \label{fig:eval-kpis}
\end{figure}

Figure~\ref{fig:eval-kpis} shows the KPI results of the RBC and SAC agents for both the training and evaluation buildings. The performance of the RBC agents is consistent across the building sets, with a slight improvement in the utilization of produced solar energy in the evaluation group. This indicates that when deploying RBC in evaluation buildings, the increase in the use of fossil fuels compared to without using the battery is similar to that of training buildings. 

On the other hand, the SAC agents of the evaluation buildings perform better than the baseline SAC agents in all KPIs except for the share of total renewable energy used. Notably, using battery storage in evaluation buildings facilitates an additional reduction of fossil energy consumption by approximately 1 \% compared to the training group.

\begin{figure}[htb]
\center
     \includegraphics[width=\textwidth]{figures/eval_results.pdf}
  \caption{}
  \label{fig:eval-results}
\end{figure}

To evaluate the effectiveness of our social method, we used a pre-trained demonstrator to train Social Agent II for the evaluation buildings. We used deterministic demonstrator actions and chose B11 and B6 as pre-trained demonstrators based on their correlation in energy consumption with the evaluation buildings. B11 has the highest median correlation without being part of the evaluation group, and B6 has the highest correlation within the group and performed the best in the training phase. We tested various imitation learning rates to determine the need for parameter tuning.

The mean value of the fossil energy consumption KPI was calculated for one experiment with and without additional policy update, as shown in Figure \ref{fig:eval-results}. Based on our analysis, we found that demonstrators B6 and B11 achieved the highest average saving of fossil energy at imitation learning rates of 0.2 and 0.3, respectively. We also discovered that the agents perform best when using demonstrator B6 with a rate of 0.25 and an additional policy update. Under these conditions, we can save about 1 \% more fossil energy than with the classic SAC agents. However, the savings achieved are lower compared to the training buildings.

  
\section{Final Discussion}
policy loss good indicator for final performance and stability of results

 Experiments mostly only one time --> more robust results if more often and than e.g. mean 
 
 MARLISA performance in paper unclear since compared to RBC, but values of RBC not given and in our case RBC worse than without battery

Social I:
Operates on policy update --> increasing value decreases the loss (not wanted), but increasing the probability more shifts the action to even more randomness. Also not exactly what we aimed. 


forwards perfect forward (but also for baseline)

pretrained Demonstrators trained on the same data of year (same weather, same fuelmix time series)

additional evaluation on new weather and consumption data of same buildings

pearson  correlation only linear
\todo[inline]{demonstrator policy update could be tried to improve more, e.g. second autotuned learning rate, other imitation learning rates, etc.}

\section{Outlook}

\subsection{Value Shaping}
Reward from Demonstrator mit einbeziehen ($-->$ Value Shaping)

In paper value function is updated, very similar to our social agent II. but since we use the absolute things there, its still frequency depending 

\subsection{Cluster Buildings}
Cluster by e.g. energy consumption or size of battery compared to consumption or PV etc etc and then use demonstrator per cluster
\cleardoublepage


%%%%%%%%%%%%%%%%%%%%%%%%%%%%%%%%%%%%%%%%%%%%%%%%%%%%%%%%%%%%%%%%%%%%%%%%%%%%%
%%% Appendix
%%%%%%%%%%%%%%%%%%%%%%%%%%%%%%%%%%%%%%%%%%%%%%%%%%%%%%%%%%%%%%%%%%%%%%%%%%%%%
\appendix

%\setcounter{secnumdepth}{-1}
%\section{Tables}\label{chap:App}
\chapter{Appendix}\label{chap:App}
\todo[inline]{only relevant ones?}
\begin{table}[htb]
\begin{tabularx}{\linewidth}{lX}
name & description \\ \hline
month & 1 (January) through 12 (December)\\
 day & type of day as provided by EnergyPlus (from 1 to 8). 1 (Sunday), 2 (Monday), ..., 7 (Saturday), 8 (Holiday)\\
 hour & hour of day (from 1 to 24)\\
 daylight\_savings\_status & indicates if the building is under daylight savings period (0 to 1). 0 indicates that the building has not changed its electricity consumption profiles due to daylight savings, while 1 indicates the period in which the building may have been affected.\\
 t\_out & outdoor temperature in Celcius degrees.\\
 t\_out\_pred\_6h & outdoor temperature predicted 6h ahead \\
 t\_out\_pred\_12h & outdoor temperature predicted 12h ahead \\
 t\_out\_pred\_24h & outdoor temperature predicted 24h ahead \\
 rh\_out & outdoor relative humidity in \%.\\
 rh\_out\_pred\_6h & outdoor relative humidity predicted 6h ahead \_\\
 rh\_out\_pred\_12h & outdoor relative humidity predicted 12h ahead \_\\
 rh\_out\_pred\_24h & outdoor relative humidity predicted 24h ahead \_\\
 diffuse\_solar\_rad & diffuse solar radiation in $W/m^2$.\\
 diffuse\_solar\_rad\_pred\_6h & diffuse solar radiation predicted 6h ahead \_\\
 diffuse\_solar\_rad\_pred\_12h & diffuse solar radiation predicted 12h ahead \_\\
 diffuse\_solar\_rad\_pred\_24h & diffuse solar radiation predicted 24h ahead \_\\
 direct\_solar\_rad & direct solar radiation in $W/m^2$. \\
 direct\_solar\_rad\_pred\_6h & direct solar radiation predicted 6h ahead \_\\ 
 direct\_solar\_rad\_pred\_12h & direct solar radiation predicted 12h ahead \_\\ 
 direct\_solar\_rad\_pred\_24h & direct solar radiation predicted 24h ahead \_\\ 
 t\_in & indoor temperature in Celcius degrees.\\ 
 avg\_unmet\_setpoint & average difference between the indoor temperatures and the cooling temperature setpoints in the different zones of the building in Celcius degrees. $sum((t\_in, t\_setpoint).clip(min=0) * zone\_volumes)/total\_volume$\\ 
 rh\_in & indoor relative humidity in \%.\\ 
 non\_shiftable\_load & electricity currently consumed by electrical appliances in kWh.\\ 
 solar\_gen & electricity currently being generated by photovoltaic panels in kWh.\\ 
 cooling\_storage\_soc & state of the charge (SOC) of the cooling storage device. From 0 (no energy stored) to 1 (at full capacity).\\ 
 dhw\_storage\_soc & state of the charge (SOC) of the domestic hot water (DHW) storage device. From 0 (no energy stored) to 1 (at full capacity).\\ 
 net\_electricity\_consumption & net electricity consumption of the building (including all energy systems) in the current time step.
\end{tabularx}
\caption[States of the CityLearn Framework]{States of the CityLearn Framework \textcolor{red}{unvollständig} \cite{vazquez2020citylearn}}\label{tab:citylearn-states}
\end{table}

\begin{table}[htb]
\begin{tabularx}{\linewidth}{lX}
name & description \\ \hline
month & 1 (January) through 12 (December)\\
day\_type & type of day as provided by EnergyPlus (from 1 to 8). 1 (Sunday), 2 (Monday), ..., 7 (Saturday), 8 (Holiday)\\
hour & hour of day (from 1 to 24)\\
 t\_out & outdoor temperature in Celcius degrees.\\
 t\_out\_pred\_6h & outdoor temperature predicted 6h ahead \textcolor{red}{(\nolinebreak accuracy: +-0.3C)}\\
 t\_out\_pred\_12h & outdoor temperature predicted 12h ahead \textcolor{red}{(\nolinebreak accuracy: +-0.65C)}\\
 t\_out\_pred\_24h & outdoor temperature predicted 24h ahead \textcolor{red}{(\nolinebreak accuracy: +-1.35C)}\\
 rh\_out & outdoor relative humidity in \%.\\
 rh\_out\_pred\_6h & outdoor relative humidity predicted 6h ahead \\
 rh\_out\_pred\_12h & outdoor relative humidity predicted 12h ahead \\
 rh\_out\_pred\_24h & outdoor relative humidity predicted 24h ahead \\
 diffuse\_solar\_rad & diffuse solar radiation in $W/m^2$.\\
 diffuse\_solar\_rad\_pred\_6h & diffuse solar radiation predicted 6h ahead \\
 diffuse\_solar\_rad\_pred\_12h & diffuse solar radiation predicted 12h ahead \\
 diffuse\_solar\_rad\_pred\_24h & diffuse solar radiation predicted 24h ahead \\
 direct\_solar\_rad & direct solar radiation in $W/m^2$. \\
 direct\_solar\_rad\_pred\_6h & direct solar radiation predicted 6h ahead \\ 
 direct\_solar\_rad\_pred\_12h & direct solar radiation predicted 12h ahead \\ 
 direct\_solar\_rad\_pred\_24h & direct solar radiation predicted 24h ahead \\ 
 wind\_speed & wind speed in $m/s$ \\
 wind\_speed\_pred\_6h & wind speed predicted 6h ahead \\
 wind\_speed\_pred\_12h & wind speed predicted 12h ahead \\
 wind\_speed\_pred\_24h & wind speed predicted 24h ahead \\
 non\_shiftable\_load & electricity currently consumed by electrical appliances in kWh.\\ 
 solar\_gen & electricity currently being generated by photovoltaic panels in kWh.\\ 
 electrical\_storage\_soc & SOC of the electrical storage from 0 (no energy stored) to 1 (at full capacity). \\
 net\_electricity\_consumption & net electricity consumption of the building (including all energy systems) in the current time step. \\
 electricity\_pricing & Electricity rate in $\$/kWh$ \\
 electricity\_pricing\_pred\_6h & Electricity rate predicted 6 hours ahead. \\
 electricity\_pricing\_pred\_12h & Electricity rate predicted 12 hours ahead. \\
 electricity\_pricing\_pred\_24h & Electricity rate predicted 24 hours ahead. \\
 renewable\_energy\_share & Share of renewable energy in the fuel mix of the grid.
\end{tabularx}
\caption[Observation space of the baseline SAC Agent]{Observation space of the baseline SAC Agent}\label{tab:citylearn-states}
\label{app:observation-space-sac}
\end{table}

%\chapter{Figures}\label{chap:App2}

\begin{table}[htb]
\begin{tabularx}{\linewidth}{lX}
Attribute Name & Description \\  \hline
Month & 1\\
Hour & 2\\ 
Day Type & 3\\
Daylight Saving Status & 4\\
Equipment Electric Power in kWh& 5\\
Solar Generation in W/kW
\end{tabularx}
\caption{\textcolor{red}{Zweite Appendix-Tabelle}}
\label{tab:buildings-vars}
\end{table}

\begin{table}[htb]
\begin{tabularx}{\linewidth}{lXXXXXXX}
Id & Demonstrator & Energy Consumption Median & Energy Consumption without storage Median & Energy Consumption without storage and PV Median \\  \hline
1 & 0  & 0.54 & 0.58 & 0.81 \\
2 & 0  & 0.45 & 0.49 & 0.74 \\
3 & 0  & 0.25 & 0.28 & 0.56 \\
4 & 0  & 0.52 & 0.54 & 0.94  \\
5 & 0  & 0.29 & 0.29 & 0.74 \\
6 & 0  & 0.5 & 0.56 & 1.08 \\
7 & 0  & 0.21 & 0.21 & 0.33 \\
8 & 0  & 0.31 & 0.32 & 0.68 \\
9 & 0  & 0.36 & 0.38 & 0.46 \\
10 & 0  & 0.56 & 0.65 & 0.98 \\
11 & 0  & 0.66 & 0.71 & 1.19 \\
12 & 0  & 0.0 & 0.0 & 0.0 \\
13 & 0  & 0.43 & 0.48 & 0.97 \\
14 & 0  & 0.49 & 0.48 & 0.59 \\
15 & 0  & 0.12 & 0.0 & 0.0 \\
16 & 0  & 0.55 & 0.72 & 1.2 \\
17 & 0  & 0.75 & 0.82 & 1.24 \\
\end{tabularx}
\caption{\textcolor{red}{Zweite Appendix-Tabelle}}
\label{tab:building-medians}
\end{table}
%\end{appendices)

\cleardoublepage

%%%%%%%%%%%%%%%%%%%%%%%%%%%%%%%%%%%%%%%%%%%%%%%%%%%%%%%%%%%%%%%%%%%%%%%%%%%%%
%%% Bibliographie
%%%%%%%%%%%%%%%%%%%%%%%%%%%%%%%%%%%%%%%%%%%%%%%%%%%%%%%%%%%%%%%%%%%%%%%%%%%%%

\addcontentsline{toc}{chapter}{Bibliography}

\bibliographystyle{alpha}
\bibliography{thesislit}
%% Obige Anweisung legt fest, dass BibTeX-Datei `mylit.bib' verwendet
%% wird. Hier koennen mehrere Dateinamen mit Kommata getrennt aufgelistet
%% werden.

\cleardoublepage
%%%%%%%%%%%%%%%%%%%%%%%%%%%%%%%%%%%%%%%%%%%%%%%%%%%%%%%%%%%%%%%%%%%%%%%%%%%%%
%%% Erklaerung
%%%%%%%%%%%%%%%%%%%%%%%%%%%%%%%%%%%%%%%%%%%%%%%%%%%%%%%%%%%%%%%%%%%%%%%%%%%%%
\thispagestyle{empty}
\section*{Selbst\"andigkeitserkl\"arung}

Hiermit versichere ich, dass ich die vorliegende Masterarbeit 
selbst\"andig und nur mit den angegebenen Hilfsmitteln angefertigt habe und dass alle Stellen, die dem Wortlaut oder dem 
Sinne nach anderen Werken entnommen sind, durch Angaben von Quellen als 
Entlehnung kenntlich gemacht worden sind. 
Diese Masterarbeit wurde in gleicher oder \"ahnlicher Form in keinem anderen 
Studiengang als Pr\"ufungsleistung vorgelegt. 

\vskip 3cm

Ort, Datum	\hfill Unterschrift \hfill 
%%%%%%%%%%%%%%%%%%%%%%%%%%%%%%%%%%%%%%%%%%%%%%%%%%%%%%%%%%%%%%%%%%%%%%%%%%%%%
%%% Ende
%%%%%%%%%%%%%%%%%%%%%%%%%%%%%%%%%%%%%%%%%%%%%%%%%%%%%%%%%%%%%%%%%%%%%%%%%%%%%

\end{document}

