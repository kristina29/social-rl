%%%%%%%%%%%%%%%%%%%%%%%%%%%%%%%%%%%%%%%%%%%%%%%%%%%%%%%%%%%%%%%%%%%%
% Soziale Agenten
%%%%%%%%%%%%%%%%%%%%%%%%%%%%%%%%%%%%%%%%%%%%%%%%%%%%%%%%%%%%%%%%%%%%
\chapter{Social Agents}
   \label{sec:social-agents}
\noindent
\todo[inline]{Ziel dieses Kapitels ist eine Einf\"uhrung in die Thematik BlaBlaBla ...}

\section{Pretrained Demonstrator}
\label{sec:pretrained-demos}
For some of our methods, we used pre-trained demonstrators. We trained these demonstrators using the final hyperparameter values, reward function and state space as described in Section \ref{sec:sac-baseline}. Also, we scaled the available renewable energy in the grid by the factor $k = 0.5$. We used two different buildings as pre-trained demonstrators, one among the training buildings and one not a training building.

\begin{figure}[htb]
\center
     \includegraphics[width=\textwidth]{figures/b5_b6_kpis.pdf}
  \caption{}
  \label{fig:b5-b6-kpis}
\end{figure}
We chose the buildings with the highest median PCC to the training buildings from the set of training buildings and the remaining ones: Building Five (training building) and Building Six. The resulting KPIs of both buildings are visualized in Figure \ref{fig:b5-b6-kpis}. The values of the share of used PV and the fossil energy consumption are somewhat comparable to the (mean) performance of the baseline SAC agents when using all training buildings. However, the average daily renewable share, both from the grid and in total, is worse when using the battery. This effect could be because the energy demand from the grid of both buildings is lower when the battery is in use since more solar energy is utilized. As a result, the proportion of renewable energy used from the grid decreases because the solar generation of the buildings is correlated to the renewable production in the grid. This, in turn, has the same effect on the overall share of renewable energy used.


\section{SAC using Demonstrator Transitions}
\todo[inline]{Font in tabellen beschreibungen verkleinern}
As mentioned in Section~\ref{sec:background-social-learning}, a popular technique to improve learning is by sampling demonstration transitions and storing them in a prioritized replay buffer (PRB). This buffer assigns priorities to transitions based on their temporal difference (TD) error. Transitions with higher TD error are given higher priority because they are expected to be more challenging to learn and should be observed more frequently during training. The transitions with higher priority are more likely to be sampled. Priorities are updated during the training process \cite{schaul2015prioritized}.

To implement the SAC agent using the demonstrator transitions, we used the trained agents previously described in Section~\ref{sec:pretrained-demos}. These agents acted on the complete dataset hourly for an entire year. We saved these transitions and then trained the agents normally, as outlined in Section~\ref{sec:sac-baseline}. The only difference was that we filled the PRB with the demonstrator transitions at the beginning of the training for each agent. We used either the transitions from Building 5 or Building 6, but not both simultaneously. The results in comparison to the baseline SAC agents are visualized in Figure \ref{fig:prb-kpis}.

\begin{figure}[htb]
\center
     \includegraphics[width=\textwidth]{figures/prb_kpis.pdf}
  \caption{}
  \label{fig:prb-kpis}
\end{figure}

The SAC agents using the demonstrator transitions perform worse than the baseline agents on all KPIs. However, using transitions from Building 5 resulted in better scores compared to the use of transitions from Building 6. This is consistent with the performance of both demonstrator buildings, as described in the previous chapter. It is possible that the better-performing Building 5 provides better guidance on how agents should behave.

The policy loss values observed during training increase for all agents instead of decreasing as expected (see Figure \ref{bla}). This observation suggests that with higher probability, actions with a low, possibly even negative, estimated Q-Value are taken. As a result, the agents are more likely to take 'bad' actions with greater certainty, ultimately leading to poorer performance.

\section{Social Agent I}
\textcolor{red}{Policy update}
First: normal policy update

Then: policy update using demonstrator actions

Classical loss $l$ to minimize is is 
\begin{align*}
	l &= \mathbb{E}_{s_t \sim \mathcal{D}, \epsilon_t \sim \mathcal{N}}[\alpha \log\pi_{\phi}(f_{\phi}(\epsilon_t;s_t)|s_t)-Q_{\theta}(s_t, f_{\phi}(\epsilon_t;s_t))] \\ 
	&=  \mathbb{E}_{s_t \sim \mathcal{D}, \epsilon_t \sim \mathcal{N}}[\alpha \mathcal{H}_{SAC}-\mathcal{V}_{SAC}] \numberthis \label{eqn:policy-loss}
\end{align*}
with the entropy term that aims to maximize randomness $\mathcal{H}_{SAC}$ and the value term that aims to maximize the estimated Q-Value of the action $\mathcal{V}_{SAC}$. For incorporating demonstrator actions similar to decision biasing described in chapter \ref{sec:decision-biasing}, so increase the probability of observed demonstrator actions or increase the estimated Q-Value of these, we modify the terms as follows: In mode 1, we use the actions sampled from the demonstrator $f^d_{\Phi}(\epsilon_t;s_t)$ in the value term and increase the estimated Q-Value by adding a fraction of the absolute value of it using the imitation learning rate $\alpha_i$:
\begin{equation}
	\mathcal{V}_{M1} =  Q_{\theta}(s_t, f_{\phi}^d(\epsilon_t;s_t)) + \alpha_i|Q_{\theta}(s_t, f_{\phi}^d(\epsilon_t;s_t))|
\end{equation}
Also, the probability of taking the demonstrator action for the demonstrator is used in the entropy term:
\begin{equation}
	\mathcal{H}_{M1} =  \log\pi_{\phi}^d(f_{\phi}^d(\epsilon_t;s_t)|s_t)
\end{equation}
In mode 2, the probability of taking the demonstrator action is increased by the learning rate:
\begin{equation}
	\mathcal{H}_{M2} = \log\pi_{\phi}^d(f_{\phi}^d(\epsilon_t;s_t)|s_t) + \alpha_i|\log\pi_{\phi}^d(f_{\phi}^d(\epsilon_t;s_t)|s_t)|
\end{equation}
Mode 3 combines mode 1 and mode 2. Mode 4-6 are similar to the modes 1-3, but the probability of taking the demonstrator action in the current trained policy is used:
\begin{equation}
	\mathcal{H}_{M4} =  \log\pi_{\phi}(f_{\phi}^d(\epsilon_t;s_t)|s_t)
\end{equation}
\begin{equation}
	\mathcal{H}_{M5} = \log\pi_{\phi}(f_{\phi}^d(\epsilon_t;s_t)|s_t) + \alpha_i|\log\pi_{\phi}(f_{\phi}^d(\epsilon_t;s_t)|s_t)|
\end{equation}
All modes calculate the loss as defined in Equation \ref{eqn:policy-loss} using the defined entropy and value term. Table \ref{tab:social-policy-update-modes} summarizes the different modes and their entropy and value terms.

{
\renewcommand{\baselinestretch}{0.9} 
\normalsize
\begin{table}[htb]
\center
\begin{tabular}{|l|l|l|}
\hline
    \textbf{Mode} & \textbf{Entropy term} & \textbf{Value term} \\
  \hline\hline
  1 & $\mathcal{H}_{M1}$ &$\mathcal{V}_{M1}$
  \\ \hline
  2 & $\mathcal{H}_{M2}$& $\mathcal{V}_{SAC}$ \\ \hline
  3 & $\mathcal{H}_{M2}$& $\mathcal{V}_{M1}$ \\ \hline 
  4 & $\mathcal{H}_{M4}$ & $\mathcal{V}_{M1}$ \\ \hline
  5 & $\mathcal{H}_{M5}$ & $\mathcal{V}_{SAC}$ \\ \hline
  6 & $\mathcal{H}_{M5}$ &  $\mathcal{V}_{M1}$ \\ \hline 
\end{tabular}
  \caption[Modes for Social Agent focusing on the policy update]{Modes for Social Agent focusing on the policy update. Mode 1 increases the estimated Q-Value of the demonstrator action, mode 2 increases the probability of the demonstrator action and mode 3 combines both. Mode 4-6 are similar, but the probability of the demonstrator actions in the current trained policy is used instead of their probabilities in the demonstrator policy.}
  \label{tab:social-policy-update-modes}
\end{table}
}


\section{Social Agent II}
Since the social policy update did not improve the performance of the models, we decided to focus on the Q-Value update in the next step. For this, we also first conducted classical Q-Value and policy updates as defined in Algorithm \ref{alg:sac-update}. Next, we performed a social Q-Value update on both Q-Value networks where we defined the target values as follows:
\begin{equation}
	Q_{\bar{\theta}}(a_t^d,s_t) + \alpha_i|Q_{\bar{\theta}}(a_t^d,s_t)| 
\end{equation}
This follows the assumption of decision biasing, that actions taken by the demonstrator $a_t^d$ are good and should thus have a higher Q-Value as proposed by \cite{najar2020actions}. Afterwards, a soft update of the target networks parameters is performed, and, optionally, a standard policy update.




