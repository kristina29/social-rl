%%%%%%%%%%%%%%%%%%%%%%%%%%%%%%%%%%%%%%%%%%%%%%%%%%%%%%%%%%%%%%%%%%%%
% Diskussion und Ausblick
%%%%%%%%%%%%%%%%%%%%%%%%%%%%%%%%%%%%%%%%%%%%%%%%%%%%%%%%%%%%%%%%%%%%

\chapter{Discussion and Outlook}
  \label{Discussion}
  
 \section{Final Discussion}

 Experiments mostly only one time --> more robust results if more often and than e.g. mean 
 
 MARLISA performance in paper unclear since compared to RBC, but values of RBC not given and in our case RBC worse than without battery

Social I:
Operates on policy update --> increasing value decreases the loss (not wanted), but increasing the probability more shifts the action to even more randomness. Also not exactly what we aimed. 

Shifted buildings --> battery capacity not changed

pretrained Demonstrators trained on the same data of year (same weather, same fuelmix time series)

pearson  correlation only linear
\todo[inline]{demonstrator policy update could be tried to improve more, e.g. second autotuned learning rate, other imitation learning rates, etc.}

\section{Outlook}

\subsection{Value Shaping}

\subsection{Cluster Buildings}
Cluster by e.g. energy consumption or size of battery compared to consumption or PV etc etc and then use demonstrator per cluster