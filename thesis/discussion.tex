%%%%%%%%%%%%%%%%%%%%%%%%%%%%%%%%%%%%%%%%%%%%%%%%%%%%%%%%%%%%%%%%%%%%
% Diskussion und Ausblick
%%%%%%%%%%%%%%%%%%%%%%%%%%%%%%%%%%%%%%%%%%%%%%%%%%%%%%%%%%%%%%%%%%%%

\chapter{Discussion and Outlook}
  \label{Discussion}
  
\section{Evaluation on new Buildings}
In the preceding sections, we presented the theoretical foundations and development of our social approach to reducing fossil energy consumption. This chapter will evaluate our method on a new set of buildings. For the assessment, we selected the six buildings 1, 2, 4, 6, 9 and 14 (see Section~\ref{sec:building-data}), which we will refer to as evaluation buildings. These buildings were not included in the training set to avoid overlap. Using the same hyperparameters, reward function, and early stopping method, we first used the asocial SAC algorithm to train each building.

\begin{figure}[htb]
\center
     \includegraphics[width=\textwidth]{figures/eval_kpis.pdf}
  \caption{}
  \label{fig:eval-kpis}
\end{figure}

Figure~\ref{fig:eval-kpis} shows the KPI results of the RBC and SAC agents for both the training and evaluation buildings. The performance of the RBC agents is consistent across the building sets, with a slight improvement in the utilization of produced solar energy in the evaluation group. This indicates that when deploying RBC in evaluation buildings, the increase in the use of fossil fuels compared to without using the battery is similar to that of training buildings. 

On the other hand, the SAC agents of the evaluation buildings perform better than the baseline SAC agents in all KPIs except for the share of total renewable energy used. Notably, using battery storage in evaluation buildings facilitates an additional reduction of fossil energy consumption by approximately 1 \% compared to the training group.

\begin{figure}[htb]
\center
     \includegraphics[width=\textwidth]{figures/eval_results.pdf}
  \caption{}
  \label{fig:eval-results}
\end{figure}

To evaluate the effectiveness of our social method, we used a pre-trained demonstrator to train Social Agent II for the evaluation buildings. We used deterministic demonstrator actions and chose B11 and B6 as pre-trained demonstrators based on their correlation in energy consumption with the evaluation buildings. B11 has the highest median correlation without being part of the evaluation group, and B6 has the highest correlation within the group and performed the best in the training phase. We tested various imitation learning rates to determine the need for parameter tuning.

The mean value of the fossil energy consumption KPI was calculated for one experiment with and without additional policy update, as shown in Figure \ref{fig:eval-results}. Based on our analysis, we found that demonstrators B6 and B11 achieved the highest average saving of fossil energy at imitation learning rates of 0.2 and 0.3, respectively. We also discovered that the agents perform best when using demonstrator B6 with a rate of 0.25 and an additional policy update. Under these conditions, we can save about 1 \% more fossil energy than with the classic SAC agents. However, the savings achieved are lower compared to the training buildings.

  
\section{Final Discussion}
policy loss good indicator for final performance and stability of results

 Experiments mostly only one time --> more robust results if more often and than e.g. mean 
 
 MARLISA performance in paper unclear since compared to RBC, but values of RBC not given and in our case RBC worse than without battery

Social I:
Operates on policy update --> increasing value decreases the loss (not wanted), but increasing the probability more shifts the action to even more randomness. Also not exactly what we aimed. 


forwards perfect forward (but also for baseline)

pretrained Demonstrators trained on the same data of year (same weather, same fuelmix time series)

additional evaluation on new weather and consumption data of same buildings

pearson  correlation only linear
\todo[inline]{demonstrator policy update could be tried to improve more, e.g. second autotuned learning rate, other imitation learning rates, etc.}

\section{Outlook}

\subsection{Value Shaping}
Reward from Demonstrator mit einbeziehen ($-->$ Value Shaping)

In paper value function is updated, very similar to our social agent II. but since we use the absolute things there, its still frequency depending 

\subsection{Cluster Buildings}
Cluster by e.g. energy consumption or size of battery compared to consumption or PV etc etc and then use demonstrator per cluster