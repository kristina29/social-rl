%%%%%%%%%%%%%%%%%%%%%%%%%%%%%%%%%%%%%%%%%%%%%%%%%%%%%%%%%%%%%%%%%%%%
% Zusammenfassung
%%%%%%%%%%%%%%%%%%%%%%%%%%%%%%%%%%%%%%%%%%%%%%%%%%%%%%%%%%%%%%%%%%%%

\chapter{Conclusion}
\label{conclusion}
\todo[inline]{What if all buildings are 'perfect' trained and will use renewable energy available in the grid --> than fossil increase again, futher investigation how that effects peak (renewable) demand}
Integrating batteries and intelligent charging is critical to reducing dependence on fossil fuels. It enables the storage of excess renewable energy during peak generation and its strategic release during periods of high demand, minimizing dependence on fossil fuels. Efficient battery management controls charging and discharging cycles to balance energy consumption with environmentally friendly practices and increase the resilience of the grid.

Our research focused on the impact of agent cooperation directed toward performance enhancement through social strategies. First, we trained standard SAC agents on a set of training buildings to establish a baseline for comparison. These agents use as information the share of renewable energy in the grid, meteorological conditions and forecasts, and energy prices that correlate with the fossil fuel mix. In addition, building-specific data such as non-shiftable load, solar generation and battery SOC were included in the training.

We developed and refined four different social strategies. The first three relied on social theories, while the fourth was an explicitly algorithmic approach.

Initially, we applied the principles of imitation learning to mimic a pre-trained demonstrator; however, neither the SAC nor the DDPG algorithm provided satisfactory results. We concluded that imitating a single demonstrator was ineffective due to the diversity of buildings, leading to the shortcomings of this method.

Subsequently, we investigated decision biasing, in which a learner tends to imitate the frequently observed actions of a demonstrator. We incorporated a social policy update into the SAC framework by modifying the entropy and value terms. Our results showed that using the demonstrator action probabilities of the learning policy was only beneficial at high imitation learning rates. However, the results were within the performance of the SAC baseline, regardless of whether random or pre-trained demonstrators were used.

Our third strategy involved a social Q-value update in SAC that positively biased the Q-value of demonstrator actions. Different demonstrators, imitation learning rates, sharing building-specific observations, and new building data were tested. The latter and also the sharing of observations showed no improvement. Using the pre-trained demonstrator B6 and its deterministic actions led to a 1.5 \% reduction in fossil energy consumption compared to the SAC agents.

In the fourth method, we used CityLearn's MARLISA algorithm to improve agent coordination through information sharing. We aimed to develop a collective reward function that outperforms the SAC baseline in reducing fossil fuel energy consumption. This approach did not meet our expectations. However, the effectiveness of MARLISA in comparison without a battery is also unclear based on the original paper.

The most effective method, which involved social Q-Value updates with deterministic actions of a pre-trained demonstrator, was then applied to a new set of evaluation buildings. The social agents resulted in an additional 1 \% saving in fossil energy compared to the baseline SAC agents. The slight decrease compared to the training buildings could be due to the different trainability of the evaluation buildings. Finding a single demonstrator that specifies helpful actions for all evaluation buildings may be more challenging than for the training buildings.

To summarize, improving the estimated Q-value for demonstrator actions may lead to better agent performance, depending on the compatibility of the demonstrator. Future studies should address the determinants of demonstrator effectiveness and experiment with different building datasets under different meteorological and energy conditions to further validate and scale our method. Other research approaches include investigating the potential of value shaping, clustering buildings by key parameters and dynamic imitation learning rates.

\clearpage
