%%%%%%%%%%%%%%%%%%%%%%%%%%%%%%%%%%%%%%%%%%%%%%%%%%%%%%%%%%%%%%%%%%%%
% Zusammenfassung
%%%%%%%%%%%%%%%%%%%%%%%%%%%%%%%%%%%%%%%%%%%%%%%%%%%%%%%%%%%%%%%%%%%%

\chapter{Conclusion}
\label{chap:conclusion}
Integrating batteries and intelligent charging is critical to reducing dependence on fossil fuels. It enables the storage of excess renewable energy during peak generation and its strategic release during periods of high demand, minimizing reliance on fossil fuels and thus reducing the carbon emissions in energy systems. Efficient battery management controls charging and discharging cycles to balance energy consumption with environmentally friendly practices and increase the resilience of the grid.

Our thesis focuses on the improvement through social strategies of multi-agent RL techniques trained to load and unload batteries. First, we train standard SAC agents on a set of training buildings to establish a baseline for comparison. These agents use as information the share of renewable energy in the grid, meteorological conditions and forecasts, and energy prices that correlate with the proportion of fossil fuels in the energy mix. In addition, building-specific data such as non-shiftable load, solar generation and battery state of charge are included in the training.

We develop and refine four different social strategies. The first three rely on social theories, while the fourth is an explicitly algorithmic approach. Initially, we apply the principles of imitation learning to mimic a pre-trained demonstrator; however, neither the SAC nor the DDPG algorithm provides satisfactory results. We conclude that imitating a single demonstrator is ineffective due to the diversity of buildings, leading to the shortcomings of this method.

Subsequently, we investigate DB, in which a learner tends to imitate the frequently observed actions of a demonstrator. We incorporate a social policy update into the SAC framework by modifying the entropy and value terms, leading to the SAC-DemoPol method. Our results show that using the demonstrator action probabilities of the learner policy is only beneficial with high ILRs. However, the results are within the performance of the SAC baseline, regardless if random or pre-trained demonstrators are used.

Our third strategy, SAC-DemoQ, involves a social Q-value update in SAC that positively biases the Q-value of demonstrator actions. Different demonstrators, ILRs, sharing building-specific observations, and new building data are tested. The latter and also the sharing of observations shows no improvement. Using the pre-trained demonstrator~D6 and its deterministic actions leads to a 1.5~\% reduction in fossil energy consumption compared with the SAC baseline agents.

In the fourth method, we use CityLearn's MARLISA algorithm to improve agent coordination through information sharing. We aim to develop a collective reward function that outperforms the SAC baseline in reducing fossil fuel energy consumption. This approach did not meet our expectations. However, the effectiveness of MARLISA in comparison without using a battery is also unclear based on the original paper.

The SAC-DemoQ method with deterministic actions of a pre-trained demonstrator shows the best results in fossil energy savings. Thus, we apply this algorithm to a new set of evaluation buildings. The social agents result in an additional 1~\% saving in fossil energy compared with the SAC baseline agents. The slight decrease compared with the training buildings could be due to the different trainability of the evaluation buildings. Finding a single demonstrator that specifies helpful actions for all evaluation buildings may be more challenging than for the training buildings.

To summarize, enhancing the estimated Q-value for demonstrator actions can improve RL agent's performances regarding fossil fuel savings, depending on the compatibility of the demonstrator. Future studies should address the determinants of demonstrator effectiveness and experiment with different building datasets under different meteorological and energy conditions to further validate and scale our method. Other research approaches include investigating the potential of VS, clustering buildings by key parameters and dynamic ILRs. Also, nuclear energy sources should be assessed separately from fossil sources to evaluate emission savings.

